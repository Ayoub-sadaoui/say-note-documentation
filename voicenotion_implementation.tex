% VoiceNotion Implementation and Development Chapter
% Chapter 3 of the VoiceNotion documentation

% --- Introduction non numérotée ---
\begin{center}
\textbf{\large Introduction du Chapitre}
\end{center}

\noindent
Ce chapitre présente en détail l'implémentation technique de VoiceNotion, en mettant l'accent sur l'architecture, les choix technologiques et les défis rencontrés lors du développement. Le lecteur découvrira comment les différentes composantes s'articulent pour offrir une expérience de prise de notes vocale innovante, performante et sécurisée.

\section{Introduction}
Cette partie du mémoire est consacrée à l'implémentation et au développement de l'application VoiceNotion. Nous allons détailler l'architecture technique, les technologies utilisées, et les différentes composantes de notre solution. VoiceNotion est une application de prise de notes vocale qui combine une interface web responsive et une application mobile, toutes deux partageant une base de données commune et des fonctionnalités similaires.

Notre approche de développement a été guidée par les principes de modularité, de réutilisabilité et d'expérience utilisateur fluide. Nous avons adopté des technologies modernes pour garantir la performance, la sécurité et l'évolutivité de notre solution.

\section{Architecture globale}
L'architecture de VoiceNotion est construite autour d'une approche multi-plateforme avec un backend commun. Cette architecture permet de maintenir une expérience utilisateur cohérente tout en optimisant le développement pour chaque plateforme.

\begin{figure}[H]
    \centering
    \includegraphics[width=0.8\textwidth]{assets/docs/golobal-diagrams/global-architecture.jpg}
    \caption{Architecture globale de VoiceNotion}
    \label{fig:global-architecture}
\end{figure}

\subsection{Composants principaux}
\begin{itemize}
    \item \textbf{Application Web}: Développée avec Next.js\resref{res:nextjs} et React, offrant une interface responsive et optimisée pour les navigateurs desktop et mobiles.
    \item \textbf{Application Mobile}: Construite avec React Native\resref{res:reactnative} et Expo\resref{res:expo}, permettant un déploiement natif sur iOS et Android.
    \item \textbf{Backend}: Utilisant Supabase\resref{res:supabase} comme solution Backend-as-a-Service (BaaS) pour l'authentification, la base de données, et le stockage.
    \item \textbf{API Gemini}: Intégration de l'API Gemini de Google pour la reconnaissance vocale et le traitement des commandes.
\end{itemize}

\subsection{Flux de données}
Le flux de données dans VoiceNotion suit un modèle client-serveur avec synchronisation en temps réel:
\begin{enumerate}
    \item L'utilisateur interagit avec l'application (web ou mobile)
    \item Les requêtes sont envoyées au backend Supabase\resref{res:supabase} via des API sécurisées
    \item Les données sont stockées dans une base de données PostgreSQL\resref{res:postgresql}
    \item Les mises à jour sont synchronisées en temps réel entre les appareils grâce aux abonnements Supabase\resref{res:supabase}
\end{enumerate}

\section{Environnement de developpement}
\subsection{Materiels}
Le developpement de VoiceNotion a ete realise sur les equipements suivants:
\begin{itemize}
    \item MacBook Pro M1 (16GB RAM, 512GB SSD)
    \item iPhone 13 Pro (pour les tests iOS)
    \item Samsung Galaxy S21 (pour les tests Android)
    \item iPad Pro (pour les tests de la version tablette)
\end{itemize}

\subsection{Logiciels et outils de developpement}

\subsubsection{Visual Studio Code}
\begin{wrapfigure}{r}{0.3\textwidth}
    \centering
    \includegraphics[width=0.25\textwidth]{assets/docs/vscode.png}
\end{wrapfigure}
Visual Studio Code est un editeur de code source leger mais puissant qui s'execute sur votre bureau et est disponible pour Windows, macOS et Linux. Il est fourni avec un support integre pour JavaScript, TypeScript\resref{res:typescript} et Node.js et dispose d'un riche ecosysteme d'extensions pour d'autres langages et environnements de developpement.

Visual Studio Code a ete notre IDE principal pour le developpement de VoiceNotion. Nous avons utilise plusieurs extensions pour ameliorer notre productivite:

\begin{itemize}
    \item ESLint\resref{res:eslint}: Pour la verification du code JavaScript/TypeScript\resref{res:typescript}
    \item Prettier\resref{res:prettier}: Pour le formatage automatique du code
    \item React Developer Tools: Pour le debogage des composants React
    \item Tailwind CSS IntelliSense\resref{res:tailwind}: Pour l'autocompletion des classes Tailwind
    \item GitLens: Pour une meilleure integration avec Git
\end{itemize}

\subsubsection{GitHub}
\begin{wrapfigure}{r}{0.3\textwidth}
    \centering
    \includegraphics[width=0.25\textwidth]{assets/docs/github.png}
\end{wrapfigure}
GitHub est une plateforme de developpement collaboratif basee sur Git, un systeme de controle de version distribue. Il est largement utilise par les developpeurs pour heberger, gerer et partager des projets de developpement de logiciels.

Sur GitHub, les developpeurs peuvent creer des depots pour stocker leur code source, collaborer avec d'autres developpeurs sur des projets, suivre et gerer les problemes, et faciliter le processus de developpement via des fonctionnalites comme les pull requests, les actions, et plus encore.

Pour VoiceNotion, nous avons utilise GitHub pour:
\begin{itemize}
    \item Gestion du code source avec branches pour chaque fonctionnalite
    \item Organisation du travail d'equipe via les issues et les projets
    \item Integration continue avec GitHub Actions\resref{res:githubactions}
    \item Revue de code via les pull requests
    \item Documentation du projet dans le wiki et le README
\end{itemize}



\subsubsection{Zod}
\begin{wrapfigure}{r}{0.3\textwidth}
    \centering
    \includegraphics[width=0.25\textwidth]{assets/docs/logo_zod.png}
\end{wrapfigure}
Zod est une bibliotheque de validation de schemas axee sur TypeScript\resref{res:typescript}. Elle offre une API puissante et expressive pour definir et valider des schemas de donnees. 

Avec Zod, nous pouvons facilement definir des regles de validation complexes pour differents types de donnees tels que les chaines de caracteres, les nombres, les tableaux, les objets, et bien d'autres. Il prend en charge des fonctionnalites avancees telles que la validation conditionnelle, les messages d'erreur personnalises et la composition de schemas.

Zod favorise le typage fort et l'inference de types, ce qui en fait un choix ideal pour les projets TypeScript\resref{res:typescript}. Il s'integre egalement parfaitement avec des frameworks et des bibliotheques populaires comme React et Express. 

Dans VoiceNotion, nous utilisons Zod pour:
\begin{itemize}
    \item Valider les donnees des formulaires utilisateur
    \item Verifier l'integrite des donnees provenant de l'API
    \item Generer des types TypeScript\resref{res:typescript} a partir des schemas de validation
    \item Assurer la coherence des donnees entre le frontend et le backend
\end{itemize}

\subsubsection{Tests}
\begin{wrapfigure}{r}{0.3\textwidth}
    \centering
    \includegraphics[width=0.25\textwidth]{assets/docs/jest.png}\\
    \vspace{0.5cm}
    \includegraphics[width=0.25\textwidth]{assets/docs/cypress.png}
\end{wrapfigure}
Pour assurer la qualite et la fiabilite de notre application, nous avons mis en place une strategie de tests complete avec Jest\resref{res:jest} et Cypress\resref{res:cypress}.

\textbf{Jest\resref{res:jest}} est un framework de test JavaScript concu pour assurer la correction de n'importe quel code JavaScript. Jest\resref{res:jest} est complet et facile a configurer, et nous l'avons utilise pour les tests unitaires et d'integration.

\textbf{Cypress\resref{res:cypress}} est un framework de test end-to-end qui nous permet de tester notre application comme le ferait un utilisateur reel. Il offre une experience de test fiable, rapide et facile a comprendre.

Notre approche de test comprend:
\begin{itemize}
    \item Tests unitaires pour les fonctions et composants individuels
    \item Tests d'integration pour verifier les interactions entre composants
    \item Tests end-to-end pour simuler les parcours utilisateur complets
    \item Tests d'accessibilite pour garantir que l'application est utilisable par tous
\end{itemize}

\section{Outils de conception et design}

\subsection{Figma}
\begin{wrapfigure}{r}{0.3\textwidth}
    \centering
    \includegraphics[width=0.25\textwidth]{assets/docs/figma.png}
\end{wrapfigure}
Figma est un outil de conception d'interface utilisateur base sur le cloud qui permet aux equipes de collaborer en temps reel. Il est devenu notre outil principal pour la conception de l'interface utilisateur et la creation de prototypes interactifs. Il nous a permis de:

\begin{itemize}
    \item Creer des wireframes et des maquettes haute fidelite
    \item Concevoir un systeme de design coherent avec des composants reutilisables
    \item Collaborer en temps reel sur les designs
    \item Tester les interactions via des prototypes cliquables
    \item Generer des specifications pour les developpeurs
\end{itemize}

L'interface intuitive de Figma et ses fonctionnalites avancees ont grandement facilite le processus de conception, permettant a notre equipe de travailler efficacement meme a distance.
\subsection{Adobe Illustrator}
\begin{wrapfigure}{r}{0.3\textwidth}
    \centering
    \includegraphics[width=0.25\textwidth]{assets/docs/illustrator.png}
\end{wrapfigure}
Adobe Illustrator est un logiciel de creation graphique et de dessin vectoriel largement utilise dans l'industrie du design. Il offre un large eventail d'outils et de fonctionnalites qui permettent de creer des illustrations, des logos, des icones, des graphiques et d'autres elements visuels de haute qualite. 

Illustrator utilise des vecteurs pour creer des images, ce qui signifie que les dessins peuvent etre redimensionnes et modifies sans perte de qualite. Il prend en charge la creation de formes, le trace de courbes, l'application de couleurs et de degrades, la manipulation des calques et bien plus encore.

Nous avons utilise Adobe Illustrator pour creer les elements graphiques vectoriels de notre identite visuelle:

\begin{itemize}
    \item Logo VoiceNotion et ses variantes
    \item Icones personnalisees
    \item Illustrations pour le site web et l'application
    \item Materiel marketing (bannieres, visuels pour reseaux sociaux)
\end{itemize}

% Explication brève avant chaque figure
\noindent
\textit{La figure suivante illustre la creation des elements graphiques avec Adobe Illustrator}
\begin{figure}[H]
\centering
\includegraphics[width=0.8\textwidth]{assets/docs/working-ulistrator.jpg}
\caption{Creation des elements graphiques avec Adobe Illustrator}
\label{fig:illustrator-assets}
\end{figure}

\subsection{Adobe Photoshop}
\begin{wrapfigure}{r}{0.3\textwidth}
    \centering
    \includegraphics[width=0.25\textwidth]{assets/docs/photoshop.png}
\end{wrapfigure}
Adobe Photoshop est un logiciel de retouche d'image et de creation graphique largement utilise dans le domaine de la conception, de la photographie et du multimedia. Il offre une gamme complete d'outils et de fonctionnalites avances pour manipuler et ameliorer les images.

Avec Photoshop, nous avons effectue des retouches precises, ajuste la luminosite et le contraste, corrige les couleurs, supprime des objets indesirables, et cree des compositions complexes pour notre application. Cet outil nous a ete particulierement utile pour:

\begin{itemize}
    \item Retoucher les captures d'ecran de l'application
    \item Creer des mockups realistes pour presentations
    \item Preparer des images optimisees pour le web et les applications mobiles
    \item Creer des elements graphiques complexes combines avec Illustrator
\end{itemize}

\section{Backend et services cloud}

\subsection{Supabase\resref{res:supabase}}
\begin{wrapfigure}{r}{0.3\textwidth}
    \centering
    \includegraphics[width=0.25\textwidth]{assets/docs/logo_supabase.png}
\end{wrapfigure}
Supabase\resref{res:supabase} est une solution Backend-as-a-Service (BaaS) open-source qui offre une alternative a Firebase. Cette plateforme nous offre:

\begin{itemize}
    \item Une base de donnees PostgreSQL\resref{res:postgresql} performante et evolutive
    \item Un systeme d'authentification securise avec plusieurs methodes de connexion
    \item Des API RESTful et GraphQL generees automatiquement
    \item Des fonctionnalites de temps reel pour la synchronisation des donnees
    \item Un stockage de fichiers integre
\end{itemize}

Nous avons choisi Supabase\resref{res:supabase} pour sa flexibilite, sa scalabilite et sa compatibilite avec PostgreSQL\resref{res:postgresql}, ce qui nous permet de beneficier d'une base de donnees relationnelle complete sans avoir a gerer l'infrastructure sous-jacente. L'authentification integree et les fonctionnalites en temps reel ont egalement considerablement accelere notre developpement.

Avec Supabase\resref{res:supabase}, nous avons pu implementer rapidement des fonctionnalites essentielles comme la synchronisation des notes entre appareils, la gestion des utilisateurs et le stockage des fichiers multimedia associes aux notes.

% Explication brève avant chaque figure
\noindent
\textit{La figure suivante illustre les fonctionnalites principales de Supabase\resref{res:supabase} pour VoiceNotion}
\begin{figure}[H]
\centering
\includegraphics[width=0.8\textwidth]{assets/docs/golobal-diagrams/supabase-feature.png}
\caption{Fonctionnalites principales de Supabase\resref{res:supabase} pour VoiceNotion}
\label{fig:supabase-feature}
\end{figure}


\subsection{Authentification avec Supabase\resref{res:supabase}}

Supabase\resref{res:supabase} fournit une solution d'authentification complète et facile à intégrer. Elle prend en charge une variété de méthodes d'authentification, y compris l'email et le mot de passe, les fournisseurs OAuth (comme Google, GitHub, etc.), et les liens magiques.

Pour VoiceNotion, nous avons utilisé l'authentification par email et mot de passe de Supabase\resref{res:supabase} pour sécuriser l'accès des utilisateurs. Le processus est simple :
\begin{itemize}
    \item \textbf{Inscription (Sign Up):} Un nouvel utilisateur crée un compte avec son adresse email et un mot de passe. Supabase\resref{res:supabase} gère la création de l'utilisateur dans sa base de données d'authentification et envoie un email de confirmation.
    \item \textbf{Connexion (Sign In):} Un utilisateur existant se connecte avec ses identifiants. Supabase\resref{res:supabase} vérifie les informations et, en cas de succès, renvoie un JSON Web Token (JWT).
    \item \textbf{Gestion de session:} Le JWT est utilisé pour authentifier les requêtes API ultérieures, permettant à l'utilisateur d'accéder aux ressources protégées. Supabase\resref{res:supabase} gère automatiquement le rafraîchissement des tokens pour maintenir la session active.
\end{itemize}

Cette approche nous a permis de mettre en place un système d'authentification robuste et sécurisé rapidement, sans avoir à gérer la complexité de l'infrastructure sous-jacente.

\begin{figure}[H]
\centering
\end{figure}


\vspace{10em} 
\captionof{figure}{Composant d'authentification}
\label{fig:auth-component}
\begin{lstlisting}[breaklines=true]
// components/AuthForm.tsx
"use client";

import { useState } from "react";
import { supabase } from "@/lib/supabase";
import { Button } from "@/components/ui/Button";
import { Input } from "@/components/ui/Input";

export default function AuthForm({ mode = "login" }) {
  const [email, setEmail] = useState("");
  const [password, setPassword] = useState("");
  const [loading, setLoading] = useState(false);
  const [error, setError] = useState("");
  
  const handleSubmit = async (e) => {
    e.preventDefault();
    setLoading(true);
    setError("");
    
    try {
      if (mode === "login") {
        const { error } = await supabase.auth.signInWithPassword({
          email,
          password,
        });
        if (error) throw error;
      } else {
        const { error } = await supabase.auth.signUp({
          email,
          password,
        });
        if (error) throw error;
      }
    } catch (error) {
      setError(error.message);
    } finally {
      setLoading(false);
    }
  };
  
  return (
    <form onSubmit={handleSubmit} className="space-y-4">
      {error && <div className="p-3 bg-red-100 text-red-700 rounded">{error}</div>}
      
      <div>
        <label htmlFor="email" className="block text-sm font-medium">
          Email
        </label>
        <Input
          id="email"
          type="email"
          value={email}
          onChange={(e) => setEmail(e.target.value)}
          required
        />
      </div>
      
      <div>
        <label htmlFor="password" className="block text-sm font-medium">
          Mot de passe
        </label>
        <Input
          id="password"
          type="password"
          value={password}
          onChange={(e) => setPassword(e.target.value)}
          required
        />
      </div>
      
      <Button type="submit" primary disabled={loading}>
        {loading 
          ? "Chargement..." 
          : mode === "login" ? "Se connecter" : "S'inscrire"}
      </Button>
    </form>
  );
}
\end{lstlisting}


\section{Application mobile}
\subsection{Technologies et outils utilises}

\subsubsection{React Native\resref{res:reactnative}}
\begin{minipage}{0.7\textwidth}
React Native\resref{res:reactnative} est un framework de developpement d'applications mobiles cree par Facebook (Meta) qui permet de construire des applications natives pour Android et iOS a partir d'une base de code JavaScript/React commune. Il offre une approche "apprendre une fois, ecrire partout" pour le developpement mobile.

Les principales caracteristiques de React Native\resref{res:reactnative} qui ont guide notre choix pour VoiceNotion sont:

\begin{itemize}
    \item \textbf{Composants natifs}: React Native\resref{res:reactnative} compile le code JavaScript en composants natifs, offrant des performances proches des applications natives
    \item \textbf{Partage de code}: Possibilite de partager une grande partie du code entre les plateformes iOS et Android
    \item \textbf{Hot Reloading}: Visualisation instantanee des modifications du code pendant le developpement
    \item \textbf{Communaute active}: Large ecosysteme de bibliotheques et support communautaire
    \item \textbf{Approche declarative}: Interface utilisateur construite de maniere declarative, similaire a React
\end{itemize}

React Native\resref{res:reactnative} nous a permis de developper l'application mobile VoiceNotion pour iOS et Android a partir d'une seule base de code, tout en offrant une experience utilisateur native et performante sur chaque plateforme.
\end{minipage}%
\hfill
\begin{minipage}{0.25\textwidth}
\centering
\includegraphics[width=0.9\textwidth]{assets/docs/logo_reactnative.png}
\end{minipage}

\begin{codebox}{Configuration du projet React Native\resref{res:reactnative}}
\begin{lstlisting}
// app.json
{
  "expo": {
    "name": "VoiceNotion",
    "slug": "voicenotion",
    "version": "1.0.0",
    "orientation": "portrait",
    "icon": "./assets/icon.png",
    "userInterfaceStyle": "light",
    "splash": {
      "image": "./assets/splash.png",
      "resizeMode": "contain",
      "backgroundColor": "#ffffff"
    },
    "assetBundlePatterns": ["**/*"],
    "ios": {
      "supportsTablet": true,
      "bundleIdentifier": "com.voicenotion.app"
    },
    "android": {
      "adaptiveIcon": {
        "foregroundImage": "./assets/adaptive-icon.png",
        "backgroundColor": "#ffffff"
      },
      "package": "com.voicenotion.app"
    },
    "plugins": [
      [
        "expo-av",
        {
          "microphonePermission": "Autoriser VoiceNotion a acceder a votre microphone."
        }
      ]
    ]
  }
}
\end{lstlisting}
\end{codebox}

\subsubsection{Expo\resref{res:expo}}
\begin{minipage}{0.7\textwidth}
Expo\resref{res:expo} est une plateforme et un ensemble d'outils construits autour de React Native\resref{res:reactnative} qui simplifient considerablement le developpement, le test et le deploiement d'applications mobiles. Il offre une approche "batteries included" pour React Native\resref{res:reactnative}.

Les avantages d'Expo\resref{res:expo} qui ont guide notre choix pour VoiceNotion sont:

\begin{itemize}
    \item \textbf{Configuration simplifiee}: Pas besoin de configurer manuellement le SDK natif d'Android ou iOS
    \item \textbf{Modules preintegres}: Acces a des APIs natives courantes (camera, geolocalisation, etc.) via des modules preconfiguress
    \item \textbf{Expo\resref{res:expo} Go}: Application de test qui permet de visualiser les changements en temps reel sur des appareils physiques
    \item \textbf{EAS (Expo\resref{res:expo} Application Services)}: Services de build, deploiement et mise a jour des applications
    \item \textbf{Over-the-air updates}: Possibilite de deployer des mises a jour sans passer par les App Stores
\end{itemize}

Expo\resref{res:expo} nous a permis d'accelerer considerablement le developpement de l'application mobile VoiceNotion, en simplifiant l'acces aux fonctionnalites natives et en offrant un flux de travail optimise du developpement au deploiement.
\end{minipage}%
\hfill
\begin{minipage}{0.25\textwidth}
\centering
\includegraphics[width=0.9\textwidth]{assets/docs/logo_expo.png}
\end{minipage}
\vspace{20em} 
\subsection{Interfaces principales}

\subsubsection{Authentification}
% Explication brève avant chaque figure
\noindent
\textit{La figure suivante illustre un aspect clé de l'architecture ou de l'implémentation technique du système.}
\begin{figure}[H]
    \centering
    \includegraphics[width=0.4\textwidth]{assets/docs/mobile/login-page.jpeg}
    \hfill
    \includegraphics[width=0.4\textwidth]{assets/docs/mobile/create-account-page.jpeg}
    \caption{Écrans de connexion et de création de compte}
    \label{fig:mobile-auth}
\end{figure}

% Explication brève avant chaque figure
\noindent
\textit{La figure suivante illustre un aspect clé de l'architecture ou de l'implémentation technique du système.}
\begin{figure}[H]
    \centering
    \includegraphics[width=0.4\textwidth]{assets/docs/mobile/forget-password-page.jpeg}
    \caption{Écran de réinitialisation du mot de passe}
    \label{fig:mobile-forgot-password}
\end{figure}

\subsubsection{Navigation et recherche}
% Explication brève avant chaque figure
\noindent
\textit{La figure suivante illustre un aspect clé de l'architecture ou de l'implémentation technique du système.}
\begin{figure}[H]
    \centering
    \includegraphics[width=0.4\textwidth]{assets/docs/mobile/home-screen.png}
    \hfill
    \includegraphics[width=0.4\textwidth]{assets/docs/mobile/search-screeen.png}
    \caption{Écrans d'accueil et de recherche}
    \label{fig:mobile-home-search}
\end{figure}

\subsubsection{Gestion des notes}
% Explication brève avant chaque figure
\noindent
\textit{La figure suivante illustre un aspect clé de l'architecture ou de l'implémentation technique du système.}
\begin{figure}[H]
    \centering
    \includegraphics[width=0.4\textwidth]{assets/docs/mobile/note-page.png}
    \hfill
    \includegraphics[width=0.4\textwidth]{assets/docs/mobile/note-page-recording.png}
    \caption{Éditeur de notes et enregistrement vocal}
    \label{fig:mobile-editor}
\end{figure}

\subsubsection{Profil utilisateur}
% Explication brève avant chaque figure
\noindent
\textit{La figure suivante illustre  un aspect clé de l'architecture ou de l'implémentation technique du système.}
\begin{figure}[H]
    \centering
    \includegraphics[width=0.4\textwidth]{assets/docs/mobile/profile-screen.png}
    \caption{Écran de profil utilisateur}
    \label{fig:mobile-profile}
\end{figure}

\section{Application web}

\subsection{Technologies et outils utilisés}

\subsubsection{Next.js\resref{res:nextjs}}
Next.js\resref{res:nextjs} est un framework open-source basé sur React, conçu pour le développement d'applications web modernes. Il permet de créer des applications rendues côté serveur (SSR), des sites statiques (SSG), et des applications web monopages (SPA) avec une expérience de développement optimisée.

Pour VoiceNotion, nous avons choisi Next.js pour ses avantages clés:
\begin{itemize}
    \item \textbf{Rendu côté serveur et statique}: Améliore les performances et le SEO.
    \item \textbf{Routing basé sur le système de fichiers}: Simplifie la navigation et l'organisation des pages.
    \item \textbf{Optimisation des images}: Composants d'image intégrés pour un chargement plus rapide.
    \item \textbf{API Routes}: Permet de créer facilement des points de terminaison d'API.
\end{itemize}

\subsubsection{TypeScript}
\begin{wrapfigure}{r}{0.3\textwidth}
    \centering
    \includegraphics[width=0.25\textwidth]{assets/docs/typescript.png}
\end{wrapfigure}
TypeScript\resref{res:typescript} est un sur-ensemble de JavaScript qui ajoute des types statiques. Il est developpe et maintenu par Microsoft.

Les avantages de TypeScript\resref{res:typescript} que nous avons exploites dans VoiceNotion:
\begin{itemize}
    \item Langage type qui permet de detecter les erreurs lors de la compilation
    \item Compilation en differentes versions ECMAScript a partir de la version 3
    \item Support de la programmation orientee objet
    \item Amelioration de la lisibilite et de la maintenance du code
\end{itemize}

\subsection{Interface de l'Application Web}
Cette section présente les différentes interfaces de l'application web VoiceNotion, illustrant le parcours de l'utilisateur, de l'authentification à la gestion des notes.

\subsubsection{Authentification}
\noindent
\textit{La figure suivante illustre un aspect clé de l'architecture ou de l'implémentation technique du système.}
\begin{figure}[H]
    \centering
    \includegraphics[width=0.45\textwidth]{assets/docs/web/login-page.png}
    \hfill
    \includegraphics[width=0.45\textwidth]{assets/docs/web/singup.png}
    \caption{Pages de connexion et d'inscription.}
    \label{fig:web-auth}
\end{figure}

\noindent
\textit{La figure suivante illustre un aspect clé de l'architecture ou de l'implémentation technique du système.}
\begin{figure}[H]
    \centering
    \includegraphics[width=0.45\textwidth]{assets/docs/web/resend-pass.png}
    \hfill
    \includegraphics[width=0.45\textwidth]{assets/docs/web/resend-succsus.png}
    \caption{Processus de réinitialisation du mot de passe.}
    \label{fig:web-reset-pass}
\end{figure}

\subsubsection{Tableau de Bord et Paramètres}
\noindent
\textit{La figure suivante illustre un aspect clé de l'architecture ou de l'implémentation technique du système.}
\begin{figure}[H]
    \centering
    \includegraphics[width=0.8\textwidth]{assets/docs/web/dashboard-page.png}
    \caption{Vue principale du tableau de bord de l'utilisateur.}
    \label{fig:web-dashboard}
\end{figure}

\noindent
\textit{La figure suivante illustre un aspect clé de l'architecture ou de l'implémentation technique du système.}
\begin{figure}[H]
    \centering
    \includegraphics[width=0.45\textwidth]{assets/docs/web/dashboard-appearence-section.png}
    \hfill
    \includegraphics[width=0.45\textwidth]{assets/docs/web/dashboard-page-security-section.png}
    \caption{Sections des paramètres : Apparence et Sécurité.}
    \label{fig:web-settings}
\end{figure}

\subsubsection{Gestion des Notes}
\noindent
\textit{La figure suivante illustre un aspect clé de l'architecture ou de l'implémentation technique du système.}
\begin{figure}[H]
    \centering
    \includegraphics[width=0.45\textwidth]{assets/docs/web/note-homepage.png}
    \hfill
    \includegraphics[width=0.45\textwidth]{assets/docs/web/note-page-dark.png}
    \caption{Page d'accueil des notes et éditeur de notes en mode sombre.}
    \label{fig:web-notes}
\end{figure}

\noindent
\textit{La figure suivante illustre un aspect clé de l'architecture ou de l'implémentation technique du système.}
\begin{figure}[H]
    \centering
    \includegraphics[width=0.45\textwidth]{assets/docs/web/on-record.png}
    \hfill
    \includegraphics[width=0.45\textwidth]{assets/docs/web/sidebar-search.png}
    \caption{Indicateur d'enregistrement et recherche dans la barre latérale.}
    \label{fig:web-record-search}
\end{figure}

\subsubsection{Publication et Partage}
\noindent
\textit{La figure suivante illustre un aspect clé de l'architecture ou de l'implémentation technique du système.}
\begin{figure}[H]
    \centering
    \includegraphics[width=0.45\textwidth]{assets/docs/web/publish-seciton.png}
    \hfill
    \includegraphics[width=0.45\textwidth]{assets/docs/web/unpublish.png}
    \caption{Options de publication et d'annulation de la publication d'une note.}
    \label{fig:web-publish}
\end{figure}

\noindent
\textit{La figure suivante illustre un aspect clé de l'architecture ou de l'implémentation technique du système.}
\begin{figure}[H]
    \centering
    \includegraphics[width=0.8\textwidth]{assets/docs/web/dashboard-public-page-dark.png}
    \caption{Exemple d'une note publiée accessible publiquement.}
    \label{fig:web-public-note}
\end{figure}




\section{Integration de la reconnaissance vocale}
\subsection{Gemini API}
\begin{minipage}{0.7\textwidth}
L'API Gemini de Google est un modele d'IA multimodal avance qui nous permet d'integrer des capacites de traitement du langage naturel et de comprehension contextuelle dans VoiceNotion. 

Cette API est au coeur de notre fonctionnalite de reconnaissance vocale et nous l'utilisons pour deux fonctions principales:
\begin{itemize}
    \item \textbf{Transcription de la parole en texte (Speech-to-Text)}: Conversion precise de l'audio vocal en texte ecrit
    \item \textbf{Analyse d'intention}: Comprehension et interpretation des commandes vocales pour executer les actions appropriees dans l'application
\end{itemize}

Gemini nous offre plusieurs avantages cles:
\begin{itemize}
    \item \textbf{Comprehension contextuelle}: Capacite a comprendre le contexte des commandes vocales
    \item \textbf{Support multilingue}: Prise en charge de multiples langues pour une utilisation internationale
    \item \textbf{Adaptabilite}: Possibilite d'affiner le modele pour notre cas d'utilisation specifique
    \item \textbf{Integration simple}: API REST facile a integrer dans nos applications web et mobile
\end{itemize}

Grace a Gemini, VoiceNotion peut offrir une experience de dictee vocale naturelle et intuitive, permettant aux utilisateurs de creer et de modifier du contenu a l'aide de commandes vocales complexes.
\end{minipage}%
\hfill
\begin{minipage}{0.25\textwidth}
\centering
\includegraphics[width=0.9\textwidth]{assets/docs/gemini.png}
\end{minipage}

% Explication brève avant chaque figure
\noindent
\textit{La figure suivante illustre le diagramme de l'API Gemini dons VoiceNotion}
\begin{figure}[H]
\centering
\includegraphics[width=0.6\textwidth]{assets/docs/golobal-diagrams/gemini-api-diagram.png}
\caption{Integration de l'API Gemini dans VoiceNotion}
\label{fig:gemini-api}
\end{figure}

\subsection{Flux de traitement des commandes vocales}
Le traitement des commandes vocales dans VoiceNotion suit un flux bien defini:
\begin{enumerate}
    \item Capture de l'audio via le microphone de l'appareil
    \item Conversion de l'audio en texte via l'API Gemini
    \item Analyse de l'intention de la commande (egalement via Gemini)
    \item Execution de l'action correspondante dans l'editeur
    \item Retour visuel et auditif a l'utilisateur
\end{enumerate}

% Explication brève avant chaque figure
\noindent
\textit{La figure suivante illustre le flux de traitement des commandes vocales dans VoiceNotion}
\begin{figure}[H]
\centering
\includegraphics[width=0.6\textwidth]{assets/docs/golobal-diagrams/flux-des-commende-vocal.png}
\caption{Flux de traitement des commandes vocales}
\label{fig:voice-commands-flow}
\end{figure}


\section{Tests et assurance qualite}
\subsection{Strategie de test}
Notre strategie de test pour VoiceNotion comprend plusieurs niveaux:
\begin{itemize}
    \item \textbf{Tests unitaires}: Verification du comportement des composants individuels
    \item \textbf{Tests d'integration}: Validation des interactions entre les differentes parties du systeme
    \item \textbf{Tests end-to-end}: Simulation des parcours utilisateur complets
    \item \textbf{Tests de performance}: Evaluation des temps de reponse et de la consommation de ressources
    \item \textbf{Tests d'accessibilite}: Verification de la conformite aux standards WCAG
\end{itemize}

\subsection{Outils de test}
\begin{itemize}
    \item \textbf{Jest\resref{res:jest}}: Framework de test pour les tests unitaires et d'integration
    \item \textbf{React Testing Library\resref{res:reacttestinglibrary}}: Bibliotheque pour tester les composants React
    \item \textbf{Cypress\resref{res:cypress}}: Outil pour les tests end-to-end de l'application web
    \item \textbf{Detox\resref{res:detox}}: Framework pour les tests end-to-end de l'application mobile
    \item \textbf{Lighthouse\resref{res:lighthouse}}: Outil pour evaluer les performances et l'accessibilite
\end{itemize}

\section{Deploiement et integration continue}
\subsection{Pipeline CI/CD}
Nous avons mis en place un pipeline d'integration continue et de deploiement continu (CI/CD) pour automatiser le processus de build, test et deploiement:

\begin{itemize}
    \item \textbf{GitHub Actions\resref{res:githubactions}}: Automatisation des workflows de CI/CD
    \item \textbf{ESLint et Prettier}: Verification de la qualite du code
    \item \textbf{Tests automatises}: Execution des tests a chaque pull request
    \item \textbf{Preview Deployments}: Deploiement de versions de previsualisation pour chaque branche
\end{itemize}

\subsection{Environnements de deploiement}
\begin{itemize}
    \item \textbf{Developpement}: Environnement local pour les developpeurs
    \item \textbf{Staging}: Environnement de preproduction pour les tests
    \item \textbf{Production}: Environnement final accessible aux utilisateurs
\end{itemize}

\subsection{Plateformes de deploiement}


% --- Deployment ---
\subsubsection{Déploiement Web et Mobile}
Pour assurer une mise en ligne simple et fiable, nous avons opté pour deux solutions de déploiement complémentaires :

\begin{minipage}{0.7\textwidth}
\textbf{Netlify\resref{res:netlify}} : le frontend web (Next.js\resref{res:nextjs}) est automatiquement buildé et publié sur Netlify\resref{res:netlify} à chaque push dans la branche \texttt{main}. Netlify\resref{res:netlify} prend en charge le CDN, le SSL/TLS et le déploiement progressif (rollback en un clic). Des variables d’environnement sécurisées y sont définies pour la connexion à Supabase\resref{res:supabase}.
\end{minipage}%
\hfill
\begin{minipage}{0.25\textwidth}
\centering
\includegraphics[width=0.9\textwidth]{assets/docs/logo_netlify.png}
\end{minipage}

\vspace{0.5cm}

\begin{minipage}{0.7\textwidth}
\textbf{Expo\resref{res:expo}} : l’application mobile React Native\resref{res:reactnative} est distribuée via la plateforme Expo\resref{res:expo}. Grâce à EAS~Build nous générons des APK/IPA signés et, via EAS~Update, nous poussons des OTA updates sans passer par les stores. Les testeurs utilisent Expo\resref{res:expo}~Go pour les versions de pré-production.
\end{minipage}%
\hfill
\begin{minipage}{0.25\textwidth}
\centering
\includegraphics[width=0.9\textwidth]{assets/docs/logo_expo.png}
\end{minipage}

Cette combinaison nous permet un cycle CI/CD unifié : GitHub Actions\resref{res:githubactions} déclenche la génération Netlify\resref{res:netlify} et les builds Expo\resref{res:expo} EAS, garantissant que chaque modification validée est rapidement disponible sur le web et sur mobile.

\section{Securite et protection des donnees}
\subsection{Authentification et autorisation}
La securite de VoiceNotion repose sur plusieurs mecanismes:
\begin{itemize}
    \item \textbf{Authentification multi-facteurs}: Pour une securite renforcee des comptes
    \item \textbf{JWT (JSON Web Tokens)}: Pour la gestion des sessions
    \item \textbf{Controle d'acces base sur les roles}: Pour limiter les actions selon le profil utilisateur
\end{itemize}

\subsection{Protection des donnees}
\begin{itemize}
    \item \textbf{Chiffrement des donnees}: En transit (HTTPS) et au repos
    \item \textbf{Anonymisation}: Des donnees utilisees pour l'analyse et l'amelioration du service
    \item \textbf{Sauvegarde reguliere}: Pour prevenir la perte de donnees
\end{itemize}

\subsection{Conformite RGPD}
VoiceNotion est concu dans le respect du Reglement General sur la Protection des Donnees:
\begin{itemize}
    \item \textbf{Consentement explicite}: Pour la collecte et l'utilisation des donnees
    \item \textbf{Droit a l'oubli}: Possibilite de supprimer toutes les donnees utilisateur
    \item \textbf{Portabilite des donnees}: Export des notes dans des formats standards
\end{itemize}

\section{Defis techniques et solutions}
\subsection{Defis rencontres}
Durant le developpement de VoiceNotion, nous avons rencontre plusieurs defis techniques:
\begin{itemize}
    \item \textbf{Precision de la reconnaissance vocale}: Particulierement dans des environnements bruyants
    \item \textbf{Synchronisation en temps reel}: Entre les differents appareils d'un utilisateur
    \item \textbf{Performance de l'editeur}: Avec des documents volumineux
    \item \textbf{Fonctionnement hors ligne}: Gestion des conflits lors de la reconnexion
\end{itemize}

\subsection{Solutions implementees}
\begin{itemize}
    \item \textbf{Filtrage audio}: Algorithmes de reduction du bruit pour ameliorer la reconnaissance vocale
    \item \textbf{Systeme de replication}: Base sur CRDT (Yjs\resref{res:crdt}) pour la synchronisation
    \item \textbf{Virtualisation}: Rendu optimise des longs documents pour maintenir la performance
    \item \textbf{Queue de synchronisation}: Pour gerer les modifications hors ligne et les synchroniser lors de la reconnexion
\end{itemize}



\section{Conclusion}
L'implementation de VoiceNotion a necessite l'integration de nombreuses technologies modernes et la resolution de defis techniques complexes. L'architecture choisie nous a permis de creer une application performante, securisee et evolutive, disponible sur le web et sur mobile.

La combinaison de React, Next.js\resref{res:nextjs}, React Native\resref{res:reactnative} et Supabase\resref{res:supabase} nous a fourni une base solide pour construire notre solution, tandis que l'integration de l'API Gemini a rendu possible la fonctionnalite centrale de reconnaissance vocale.

Les prochaines etapes de developpement incluront l'amelioration continue de la precision de la reconnaissance vocale, l'ajout de nouvelles fonctionnalites collaboratives, et l'optimisation des performances sur tous les appareils. 

% --- Conclusion non numérotée ---
\vspace{1cm}
\begin{center}
\textbf{\large Conclusion du Chapitre}
\end{center}

\noindent
En résumé, ce chapitre a détaillé l'ensemble des choix techniques, des architectures et des solutions mises en œuvre pour concrétiser VoiceNotion. L'intégration harmonieuse des différentes technologies et la résolution des défis techniques témoignent de la robustesse et de la modernité de l'application. Ces fondations solides ouvrent la voie à de futures évolutions et améliorations, qui seront abordées dans les perspectives finales du document.