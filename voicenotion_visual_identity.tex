\section{VoiceNotion – Brand Identity Snapshot}

\subsection{Brand Essence}
\begin{quote}
    “A voice-first productivity companion that brings calm, clarity, and a spark of joy to capturing and organizing your thoughts.”
\end{quote}

\subsection{Core Brand Personality (4 Traits)}
\begin{tabular}{|l|l|}
\hline
\textbf{Trait} & \textbf{Description} \\
\hline
Calm & Minimal, distraction-free, focused like Notion. \\
\hline
Clear & Straightforward and precise—no clutter, no fluff. \\
\hline
Supportive & Friendly, helpful, and empowering—always welcoming. \\
\hline
Lightly Joyful & Adds subtle positivity and delight—never too serious. \\
\hline
\end{tabular}

\subsection{Positioning Pillars}
\begin{tabular}{|l|l|}
\hline
\textbf{Pillar} & \textbf{VoiceNotion Delivers} \\
\hline
100\% Voice-First & Speak to write, format, and organize—hands-free productivity \\
\hline
AI-Powered Clarity & Smart transcription, summaries, and auto-tagging \\
\hline
Calm \& Focused UX & Thoughtful, minimal interface to reduce friction \\
\hline
Inclusive Design & Accessibility-first, multilingual, voice-only navigation \\
\hline
\end{tabular}

\subsection{Key Differentiators}
\begin{tabular}{|l|l|l|}
\hline
\textbf{Feature} & \textbf{Others} & \textbf{VoiceNotion} \\
\hline
Fully voice-operated & \ding{55} & \ding{51} \\
\hline
No typing required & \ding{55} & \ding{51} \\
\hline
AI summaries + tagging & \ding{51} & \ding{51} \\
\hline
Voice-controlled workspace & \ding{55} & \ding{51} \\
\hline
Accessibility-first UX & \ding{55} & \ding{51} \\
\hline
\end{tabular}

\subsection{Brand Voice Principles}
\begin{itemize}
    \item \textbf{Calm \& Clear}: “Start speaking and I’ll take it from here.”
    \item \textbf{Supportive \& Positive}: “Got it! Note saved.”
    \item \textbf{Simple, No Jargon}: “Your voice. Your notes. No keyboard.”
    \item \textbf{No Talking Back}: Text-based feedback only. Quiet UX.
\end{itemize}

\section{Visual Identity System – VoiceNotion}

\subsection{Accent Color Suggestion}
\subsubsection{Soft Indigo (\#5C6AC4)}
\begin{itemize}
    \item \textbf{Why?}: Indigo has a calm, thoughtful energy, but this soft variant adds a \textbf{modern, focused, yet gently vibrant} feel.
    \item It’s professional \textbf{without being too corporate}, and joyful \textbf{without being childish}.
    \item Also works beautifully with dark mode AND light mode.
\end{itemize}

\subsubsection{Alternatives (in same vibe):}
\begin{itemize}
    \item \textbf{Slate Blue} \#6A7FDB – more techy \& cool
    \item \textbf{Muted Orchid} \#A88FBD – adds subtle joy + uniqueness
    \item \textbf{Cool Mint} \#9EEBCF – fresh, accessible, voice-inspired
\end{itemize}

% Placeholder for Color Palette Image
% \begin{figure}[h!]
%     \centering
%     \includegraphics[width=0.8\textwidth]{assets/docs/colors/color_palette.png}
%     \caption{VoiceNotion Color Palette}
%     \label{fig:color_palette}
% \end{figure}

\subsection{Font Suggestions}
You want something clean, modern, and human — just like your brand tone.
\begin{tabular}{|l|l|l|}
\hline
\textbf{Font} & \textbf{Style} & \textbf{Why It Fits} \\
\hline
\textbf{Inter} & Sans-serif, modern & Designed for interfaces. Clean, readable, calm vibe. \\
\hline
\textbf{General Sans} & Friendly sans-serif & Slightly rounded for a friendly, clear feel. \\
\hline
\textbf{Outfit} & Geometric, simple & Soft and techy. Great for voice interfaces. \\
\hline
\textbf{Satoshi} & Neutral + modern & Balanced between formal and casual. Great for clarity. \\
\hline
\end{tabular}

\textit{Avoid overly playful fonts like Comic Neue or too serious fonts like Times New Roman.}

\subsection{Visual Elements and Style}

\subsubsection{Iconography}
\begin{itemize}
    \item Use \textbf{thin-line, rounded icons} (e.g., Lucide, Feather Icons)
    \item Avoid overly detailed or harsh outlines
\end{itemize}

\subsubsection{Shapes}
\begin{itemize}
    \item Rounded corners (8px–16px) for buttons, cards, inputs
    \item Fluid shapes to suggest voice flow (like audio waves, soft curves)
\end{itemize}

\subsubsection{Imagery \& Illustrations}
\begin{itemize}
    \item Use \textbf{abstract voice waveforms}, \textbf{flow lines}, and \textbf{floating notes}
    \item Keep imagery light and clean (no overly literal or corporate stock)
\end{itemize}

\subsubsection{Themes}
\begin{itemize}
    \item \textbf{Light Mode}: Soft off-white background (\#FAFAFC)
    \item \textbf{Dark Mode}: Deep indigo/blue-gray (\#1E1E2F) with your accent color glowing subtly
\end{itemize}

\subsection{Summary Visual System}
\begin{tabular}{|l|l|}
\hline
\textbf{Element} & \textbf{Design Direction} \\
\hline
\textbf{Accent Color} & Soft Indigo \#5C6AC4 \\
\hline
\textbf{Typography} & Inter / General Sans / Outfit \\
\hline
\textbf{UI Style} & Minimal, rounded, lightly playful \\
\hline
\textbf{Icon Style} & Line-based, round corners, consistent weight \\
\hline
\textbf{Backgrounds} & Light: \#FAFAFC / Dark: \#1E1E2F \\
\hline
\textbf{Mood} & Calm, voice-centric, modern, slightly joyful \\
\hline
\end{tabular}
