\chapter{IDENTITÉ VISUELLE}

\section{Introduction}

Ce chapitre présente l'identité visuelle de VoiceNotion, un élément clé pour créer une expérience utilisateur cohérente et mémorable. L'identité visuelle traduit la personnalité de la marque en éléments graphiques concrets, allant des couleurs et des polices aux icônes et à l'imagerie. Une identité visuelle forte et bien définie est essentielle pour différencier VoiceNotion de ses concurrents, renforcer la reconnaissance de la marque et créer un lien émotionnel avec les utilisateurs.

L'objectif de cette identité visuelle est de refléter les valeurs fondamentales de VoiceNotion : la clarté, le calme, le soutien et une touche de joie. Chaque élément a été choisi pour contribuer à une expérience utilisateur fluide, intuitive et agréable, en particulier dans le contexte d'une application axée sur la voix.

\section{Instantané de l'Identité de Marque VoiceNotion}

\subsection{Essence de la Marque}

\begin{quote}
    “Un compagnon de productivité axé sur la voix qui apporte calme, clarté et une étincelle de joie à la capture et à l'organisation de vos pensées.”
\end{quote}

\subsection{Personnalité de la Marque (4 Traits)}

\begin{table}[H]
    \centering
    \begin{tabular}{|p{4cm}|p{9cm}|}
        \hline
        \textbf{Trait} & \textbf{Description} \\
        \hline
        \textbf{Calme} & Minimal, sans distraction, concentré comme Notion. \\
        \hline
        \textbf{Clair} & Direct et précis—sans encombrement, sans superflu. \\
        \hline
        \textbf{Soutenant} & Amical, serviable et responsabilisant—toujours accueillant. \\
        \hline
        \textbf{Légèrement Joyeux} & Ajoute une positivité et un plaisir subtils—jamais trop sérieux. \\
        \hline
    \end{tabular}
    \caption{Personnalité de la marque VoiceNotion}
    \label{tab:brand_personality}
\end{table}

\subsection{Piliers de Positionnement}

\begin{table}[H]
    \centering
    \begin{tabular}{|p{4cm}|p{9cm}|}
        \hline
        \textbf{Pilier} & \textbf{Ce que VoiceNotion Apporte} \\
        \hline
        \textbf{100\% Axé sur la Voix} & Parlez pour écrire, formater et organiser—productivité mains libres. \\
        \hline
        \textbf{Clarté Alimentée par l'IA} & Transcription intelligente, résumés et étiquetage automatique. \\
        \hline
        \textbf{UX Calme et Concentrée} & Interface réfléchie et minimale pour réduire les frictions. \\
        \hline
        \textbf{Conception Inclusive} & Accessibilité d'abord, multilingue, navigation uniquement par la voix. \\
        \hline
    \end{tabular}
    \caption{Piliers de positionnement de VoiceNotion}
    \label{tab:positioning_pillars}
\end{table}

\subsection{Principaux Différenciateurs}

\begin{table}[H]
    \centering
    \begin{tabular}{|p{5cm}|c|c|}
        \hline
        \textbf{Fonctionnalité} & \textbf{Autres} & \textbf{VoiceNotion} \\
        \hline
        Entièrement commandé par la voix & \textcolor{red}{Non} & \textcolor{green!60!black}{Oui} \\
        \hline
        Aucune saisie requise & \textcolor{red}{Non} & \textcolor{green!60!black}{Oui} \\
        \hline
        Résumés IA + étiquetage & \textcolor{green!60!black}{Oui} & \textcolor{green!60!black}{Oui} \\
        \hline
        Espace de travail contrôlé par la voix & \textcolor{red}{Non} & \textcolor{green!60!black}{Oui} \\
        \hline
        UX axée sur l'accessibilité & \textcolor{red}{Non} & \textcolor{green!60!black}{Oui} \\
        \hline
    \end{tabular}
    \caption{Principaux différenciateurs de VoiceNotion}
    \label{tab:key_differentiators}
\end{table}

\subsection{Principes de la Voix de la Marque}

\begin{itemize}
    \item \textbf{Calme et Clair :} “Commencez à parler et je m'occupe du reste.”
    \item \textbf{Soutenant et Positif :} “Compris ! Note enregistrée.”
    \item \textbf{Simple, sans Jargon :} “Votre voix. Vos notes. Sans clavier.”
    \item \textbf{Pas de Réponse Vocale :} Feedback textuel uniquement. UX silencieuse.
\end{itemize}

\section{Système d'Identité Visuelle – VoiceNotion}

\subsection{Suggestion de Couleur d'Accentuation}

\subsubsection{Indigo Doux (\texttt{\#5C6AC4})}

\begin{itemize}
    \item \textbf{Pourquoi ?} L'indigo a une énergie calme et réfléchie, mais cette variante douce ajoute une sensation \textbf{moderne, concentrée, mais légèrement vibrante}.
    \item C'est professionnel \textbf{sans être trop corporatif}, et joyeux \textbf{sans être enfantin}.
    \item Fonctionne aussi magnifiquement avec le mode sombre ET le mode clair.
\end{itemize}

\subsubsection{Alternatives (dans la même ambiance)}

\begin{itemize}
    \item \textbf{Bleu Ardoise} \texttt{\#6A7FDB} – plus technique et cool
    \item \textbf{Orchidée Discrète} \texttt{\#A88FBD} – ajoute une joie et une unicité subtiles
    \item \textbf{Menthe Fraîche} \texttt{\#9EEBCF} – frais, accessible, inspiré par la voix
\end{itemize}

\subsection{Suggestions de Polices}

Nous voulons quelque chose de propre, moderne et humain — tout comme le ton de notre marque.

\begin{table}[H]
    \centering
    \begin{tabular}{|p{3cm}|p{3cm}|p{7cm}|}
        \hline
        \textbf{Police} & \textbf{Style} & \textbf{Pourquoi ça correspond} \\
        \hline
        \textbf{Inter} & Sans-serif, moderne & Conçu pour les interfaces. Propre, lisible, ambiance calme. \\
        \hline
        \textbf{General Sans} & Sans-serif amical & Légèrement arrondi pour une sensation amicale et claire. \\
        \hline
        \textbf{Outfit} & Géométrique, simple & Doux et technique. Idéal pour les interfaces vocales. \\
        \hline
        \textbf{Satoshi} & Neutre + moderne & Équilibré entre formel et décontracté. Idéal pour la clarté. \\
        \hline
    \end{tabular}
    \caption{Suggestions de polices pour VoiceNotion}
    \label{tab:font_suggestions}
\end{table}

\textit{Évitez les polices trop ludiques comme Comic Neue ou trop sérieuses comme Times New Roman.}

\subsection{Éléments Visuels et Style}

\subsubsection{Iconographie}

\begin{itemize}
    \item Utilisez des \textbf{icônes fines et arrondies} (par ex., Lucide, Feather Icons)
    \item Évitez les contours trop détaillés ou durs
\end{itemize}

\subsubsection{Formes}

\begin{itemize}
    \item Coins arrondis (8px–16px) pour les boutons, cartes, entrées
    \item Formes fluides pour suggérer le flux de la voix (comme des ondes audio, des courbes douces)
\end{itemize}

\subsubsection{Imagerie et Illustrations}

\begin{itemize}
    \item Utilisez des \textbf{ondes vocales abstraites}, des \textbf{lignes de flux} et des \textbf{notes flottantes}
    \item Gardez l'imagerie légère et propre (pas d'images de stock trop littérales ou corporatives)
\end{itemize}

\subsubsection{Thèmes}

\begin{itemize}
    \item \textbf{Mode Clair :} Fond blanc cassé doux (\texttt{\#FAFAFC})
    \item \textbf{Mode Sombre :} Indigo profond/gris-bleu (\texttt{\#1E1E2F}) avec votre couleur d'accentuation brillant subtilement
\end{itemize}

\subsection{Résumé du Système Visuel}

\begin{table}[H]
    \centering
    \begin{tabular}{|p{4cm}|p{9cm}|}
        \hline
        \textbf{Élément} & \textbf{Direction de Conception} \\
        \hline
        \textbf{Couleur d'Accentuation} & Indigo Doux \texttt{\#5C6AC4} \\
        \hline
        \textbf{Typographie} & Inter / General Sans / Outfit \\
        \hline
        \textbf{Style UI} & Minimal, arrondi, légèrement ludique \\
        \hline
        \textbf{Style d'Icône} & Basé sur des lignes, coins ronds, poids constant \\
        \hline
        \textbf{Arrière-plans} & Clair : \texttt{\#FAFAFC} / Sombre : \texttt{\#1E1E2F} \\
        \hline
        \textbf{Ambiance} & Calme, centré sur la voix, moderne, légèrement joyeux \\
        \hline
    \end{tabular}
    \caption{Résumé du système visuel de VoiceNotion}
    \label{tab:visual_system_summary}
\end{table}
