% ===============================
% SayNote Documentation (Supervisor-Compliant Structure)
% ===============================
\documentclass[12pt,a4paper]{report}

% --- Packages ---
\usepackage[utf8]{inputenc}
\usepackage[T1]{fontenc}
\usepackage[french,english]{babel}
\usepackage{graphicx}
\usepackage{float}
\usepackage{hyperref}
\usepackage{enumitem}
% Disable red link borders in contents/figures
\hypersetup{
  hidelinks,
  colorlinks=true,
  linkcolor=black,
  urlcolor=blue,
  citecolor=black
}
\usepackage{geometry}
\usepackage{fancyhdr}
\usepackage[listings]{tcolorbox}
\usepackage{xcolor}
\usepackage{wrapfig}
\usepackage{tikz}
\usepackage{listings}
\usepackage{caption}
\definecolor{SayNoteLightGray}{RGB}{240,240,240}
% Dark background color for code snippets
\definecolor{codebg}{RGB}{30,30,30}
% Global style for all listings
% macros
\newcommand{\resref}[1]{\hyperref[#1]{[\ref*{#1}]}}

\lstset{
  basicstyle=\ttfamily\footnotesize\color{white},
  backgroundcolor=\color{codebg},
  keywordstyle=\color{cyan},
  commentstyle=\color{green!70},
  stringstyle=\color{yellow},
  numbers=left,
  numberstyle=\tiny\color{gray!50},
  breaklines=true,
  columns=fullflexible
}

% Define a simple codebox environment for code snippets
\newtcblisting{codebox}{%
  listing only,
  colback=codebg,
  colframe=codebg!80,
  left=1mm, right=1mm, top=1mm, bottom=1mm,
  boxrule=0.5pt, arc=2pt,
  fonttitle=\bfseries,
  title=Code,
  listing options={%
    basicstyle=\ttfamily\footnotesize\color{white},
    breaklines=true,
    columns=fullflexible,
    keywordstyle=\color{cyan},
    commentstyle=\color{green!70},
    stringstyle=\color{yellow},
    numbers=left,
    numberstyle=\tiny\color{gray!50},
    backgroundcolor=\color{codebg}
  }
}
% Configure table of contents depth and numbering
\setcounter{tocdepth}{3}
\setcounter{secnumdepth}{3}

\geometry{margin=2.5cm}

% --- Title ---
\title{SayNote: Documentation}
\author{BENBOUZID Rima, DOUMAZ Imene}
\date{\today}

\definecolor{coverBorderColor}{RGB}{0,0,0}
\usetikzlibrary{calc}
\begin{document}

% --- Cover Page ---
% \maketitle

% --- Custom Original Cover Page ---
\begin{titlepage}
\thispagestyle{empty}

\begin{tikzpicture}[remember picture,overlay]
  % Draw thick black border around the page with reduced top/bottom margins
  \draw [line width=2pt, coverBorderColor] 
    ($(current page.north west) + (0.8cm,-0.4cm)$) 
    rectangle 
    ($(current page.south east) + (-0.8cm,0.4cm)$);
    
  % Draw inner border - thinner line with reduced top/bottom margins
  \draw [line width=0.5pt, coverBorderColor] 
    ($(current page.north west) + (1.1cm,-0.7cm)$) 
    rectangle 
    ($(current page.south east) + (-1.1cm,0.7cm)$);
\end{tikzpicture}

\begin{center}
    \vspace*{0.1cm}
    {\Large \textbf{République Algérienne Démocratique et Populaire}} \\
    \vspace{0.3cm}
    {\large Ministère de l'Enseignement Supérieur et de la Recherche Scientifique} \\
    \vspace{0.3cm}
    {\large Université M'hamed Bougara - Boumerdès} \\
    \vspace{0.6cm}
    
    \includegraphics[width=4.5cm]{assets/docs/university_logo.png} \\
    \vspace{0.6cm}
    
    {\large Faculté des Sciences} \\
    \vspace{0.2cm}
    {\large Département d'Informatique} \\
    \vspace{0.6cm}
    
    \begin{tabular}{rl}
        \textbf{Domaine} & : Mathématiques Informatique \\
        \textbf{Filière} & : Informatique \\
        \textbf{Spécialité} & : Développement Web et Infographie \\
    \end{tabular} \\
    
    {\small \textit{N° de l'Arrêté d'habilitation de la spécialité : arrêté n°002 du 03/01/2021}} \\
    \vspace{0.7cm}
    
    {\large \textit{\underline{Mémoire de fin d'études en vu de l'obtention du}}} \\
    {\large \textit{\underline{Diplôme de Licence Professionnelle}}} \\
    \vspace{0.7cm}
    
    {\LARGE \textbf{Thème}} \\
    \vspace{0.6cm}
    
    % Create box with project title matching the reference image
    \begin{tikzpicture}
        \node[draw, rounded corners=15pt, line width=1pt, minimum width=14cm, minimum height=3.5cm] {
            \begin{minipage}{13cm}
                \begin{center}
                    \vspace{-0.5cm}
                    \fontsize{22}{26}\selectfont \textbf{SayNote} \\
                    \fontsize{22}{26}\selectfont \textbf{AI-Based Voice Productivity SaaS Application}
                \end{center}
            \end{minipage}
        };
    \end{tikzpicture}
    \vspace{0.8cm}

    \begin{tabular}{p{8cm}p{6cm}}
        \textbf{Présenté par :} & \textbf{Supervisé par :} \\
        BENHADJER Foudhil Abdellah & CHAOUCHE Ali \\
        SADAOUI Ayoub & \\
        HAFSAOUI Abderahim & \\
    \end{tabular} \\
    \vspace{0.8cm}
    
    {\normalsize Soutenu le \underline{24/06/2024} Devant le jury composé de} \\
    \vspace{0.4cm}
    \begin{tabular}{lll}
        HAMICHE Mokhtar & : & Examinateur \\
        BELKASMI Djamel & : & Examinateur \\
        YAHIATENE Youcef & : & Président \\
    \end{tabular}
\end{center}
\end{titlepage}


% --- Résumé & Abstract ---
% ===============================
% Résumé (French Summary)
% ===============================
\chapter*{Résumé}
\addcontentsline{toc}{chapter}{Résumé}
Ce mémoire présente la conception et le développement de SayNote, une application innovante de prise de notes basée sur la reconnaissance vocale et l’édition par blocs. L’objectif principal est de permettre aux utilisateurs de capturer, organiser et structurer rapidement leurs idées à l’aide de commandes vocales intuitives, tout en offrant une expérience utilisateur moderne et multiplateforme.

Après une étude approfondie des méthodes traditionnelles et des solutions existantes, nous proposons une architecture technique fondée sur les technologies web et mobiles les plus récentes, intégrant des fonctionnalités avancées telles que la transcription vocale intelligente, la gestion flexible des notes et une interface utilisateur épurée. Le projet met également l’accent sur l’accessibilité, la sécurité des données et l’intégration d’une identité visuelle forte.

Les résultats obtenus démontrent la faisabilité et la pertinence de SayNote pour répondre aux besoins actuels des étudiants, professionnels et créateurs de contenu. Ce travail ouvre la voie à de futures améliorations, notamment l’enrichissement des fonctionnalités collaboratives et l’optimisation continue de la reconnaissance vocale.

% ===============================
% Abstract (English Summary)
% ===============================
\chapter*{Abstract}
\addcontentsline{toc}{chapter}{Abstract}
This thesis presents the design and development of VoiceNotion, an innovative note-taking application based on voice recognition and block-based editing. The main objective is to enable users to quickly capture, organize, and structure their ideas using intuitive voice commands, while providing a modern, cross-platform user experience.

After a thorough study of traditional methods and existing solutions, we propose a technical architecture built on the latest web and mobile technologies, integrating advanced features such as intelligent voice transcription, flexible note management, and a streamlined user interface. The project also emphasizes accessibility, data security, and the integration of a strong visual identity.

The results demonstrate the feasibility and relevance of VoiceNotion in meeting the current needs of students, professionals, and content creators. This work paves the way for future improvements, including enhanced collaborative features and ongoing optimization of voice recognition.


% --- Table of Contents, Figures, Tables ---
\tableofcontents
\listoffigures
% \listoftables

% --- General Introduction ---
\chapter*{Introduction générale}
\addcontentsline{toc}{chapter}{Introduction générale}
% Introduction Chapter for SayNote Documentation

% --- Introduction non numérotée ---
\begin{center}
\textbf{\large Introduction du Chapitre}
\end{center}

\noindent
Ce chapitre présente le contexte, les objectifs et la portée du projet SayNote. Il offre une vue d'ensemble sur la motivation derrière l'application, les principales fonctionnalités attendues, ainsi que la structure du document. Le lecteur acquerra une compréhension globale du projet avant d'entrer dans les détails techniques et conceptuels des chapitres suivants.

\thispagestyle{fancy}

\vspace{1cm}

Dans un monde en constante évolution, la technologie est devenue essentielle pour transformer divers secteurs, y compris le domaine de la prise de notes et de la création de documents. Avec l'essor des assistants vocaux et des technologies de reconnaissance vocale, le potentiel d'innovation dans la façon dont nous capturons et organisons nos pensées est immense.

SayNote représente une approche innovante pour capturer, organiser et affiner les pensées principalement par commandes vocales, complétée par un éditeur intuitif basé sur des blocs. L'application vise à être la solution de référence pour les utilisateurs qui valorisent la rapidité, l'efficacité et la flexibilité de la saisie vocale, sans sacrifier les riches capacités d'édition des éditeurs de blocs modernes.

\vspace{0.5cm}

L'idée principale derrière SayNote est de créer une expérience fluide de prise de notes et de création de documents, optimisée pour les appareils mobiles. L'application s'adresse aux étudiants, aux professionnels, aux écrivains et à toute personne ayant besoin de noter rapidement des idées, d'organiser des notes ou de rédiger des documents en déplacement.

\vspace{0.5cm}

La vision de SayNote est de devenir l'application de choix pour ceux qui cherchent à maximiser leur productivité en transformant la parole en contenu structuré et organisé. En combinant la puissance des commandes vocales avec l'organisation intuitive des éditeurs de blocs, SayNote offre une solution innovante aux défis de la prise de notes traditionnelle.

\vspace{1cm}

\section*{Objectifs du projet}
\addcontentsline{toc}{section}{Objectifs du projet}

Les objectifs principaux du projet SayNote sont :

\begin{itemize}
    \item Développer une application mobile entièrement fonctionnelle permettant aux utilisateurs de créer et d'éditer des notes via des commandes vocales.
    \item Implémenter un éditeur basé sur BlockNote.js offrant une expérience d'édition par blocs similaire à Notion.
    \item Assurer une transcription vocale de haute fidélité et une analyse intelligente des intentions de l'utilisateur.
    \item Créer une interface utilisateur intuitive et conviviale optimisée pour les appareils mobiles.
    \item Fournir une application web complémentaire pour une expérience cross-platform complète.
\end{itemize}

\vspace{1cm}

\section*{Portée du document}
\addcontentsline{toc}{section}{Portée du document}

Ce document sert de guide complet pour l'application SayNote, couvrant sa base conceptuelle, son architecture technique, les détails d'implémentation, la conception de l'expérience utilisateur et l'identité visuelle. Il est destiné aux développeurs, designers et parties prenantes impliqués dans le projet, fournissant une compréhension approfondie de la structure et de la fonctionnalité de l'application.

La documentation est organisée en plusieurs chapitres, chacun couvrant un aspect spécifique du développement et de la conception de SayNote :

\begin{itemize}
    \item \textbf{Étude préalable} : Analyse du problème, objectifs du projet et solutions proposées.
    \item \textbf{Planification et conception UX} : Méthodologie de développement, recherche utilisateur et conception du système.
    \item \textbf{Implémentation et développement} : Architecture technique, technologies utilisées et détails d'implémentation.
    \item \textbf{Identité visuelle} : Conception de la marque, éléments visuels et interfaces utilisateur.
\end{itemize}

Ce document servira de référence tout au long du cycle de développement du projet et pourra être utilisé comme base pour la formation, la maintenance et l'évolution future de l'application SayNote. 

% --- Conclusion non numérotée ---
\vspace{1cm}
\begin{center}
\textbf{\large Conclusion du Chapitre}
\end{center}

\noindent
En résumé, ce chapitre introductif a permis de cerner le contexte, les objectifs et la portée du projet SayNote. Il prépare le lecteur à explorer les aspects détaillés de la solution, depuis l'étude préalable jusqu'à la conception, l'implémentation et l'identité visuelle, en fournissant une base solide pour la compréhension du reste de la documentation.

% ===============================
% PART I (unnamed): State of the Art
% ===============================

% --- Chapter 1: Study of the existence ---
% --- PART I: State of the Art ---
\cleardoublepage
\part*{Part I : State of the Art}
\addcontentsline{toc}{part}{Part I : State of the Art}
\thispagestyle{empty}

\chapter{Étude de l'existant}
% SayNote - Chapitre 1 : Étude de l’existant
% Ce chapitre analyse l’état actuel de la prise de notes et de la gestion d’informations, en s’appuyant sur les pratiques, outils et défis rencontrés par les utilisateurs.

\section*{Introduction}
\addcontentsline{toc}{section}{Introduction}
Ce chapitre dresse un état des lieux des pratiques actuelles de prise de notes et de gestion d’informations. Il identifie les méthodes utilisées, les difficultés rencontrées par les utilisateurs, et les limites des solutions existantes. L’objectif est de comprendre le contexte et les besoins réels des utilisateurs afin de mettre en lumière les défis à relever, constituant ainsi le socle de notre réflexion pour une solution innovante.

\section{Contexte de la prise de notes}

La prise de notes intervient dans des contextes variés : réunions, cours, brainstorming, déplacements, etc. Les utilisateurs cherchent à capturer rapidement des idées, à structurer des informations, et à pouvoir y accéder facilement sur différents supports (papier, ordinateur, mobile). L’évolution des usages montre une transition progressive du papier vers les outils numériques, sans pour autant résoudre tous les problèmes liés à la gestion efficace des notes.

\section{Méthodes actuelles de gestion des notes et informations}

La gestion des notes repose aujourd’hui sur une grande diversité de méthodes et d’outils, qu’on peut regrouper en plusieurs catégories :

\subsection{Prise de notes manuelle}
\begin{itemize}
    \item Utilisation de carnets, feuilles volantes, post-its.
    \item Facilité d’accès mais risque élevé de perte, de désorganisation ou d’oubli.
    \item Difficulté à rechercher ou à partager l’information.
\end{itemize}

\subsection{Outils numériques classiques}
\begin{itemize}
    \item Logiciels de traitement de texte (Word, Google Docs), tableurs, emails.
    \item Avantages : sauvegarde, partage, édition facile.
    \item Limites : structure peu adaptée à la prise de notes rapide, fragmentation des fichiers, manque de synchronisation multi-appareils.
\end{itemize}

\subsection{Applications spécialisées de prise de notes}
\begin{itemize}
    \item Applications comme Notion, Evernote, OneNote, Google Keep, Apple Notes.
    \item Fonctions avancées : organisation hiérarchique, recherche, étiquettes, synchronisation cloud, prise de notes vocale.
    \item Limitations relevées par les utilisateurs :
    \begin{itemize}
        \item Transcription vocale parfois imprécise, surtout en environnement bruyant ou multilingue.
        \item Fonctionnalités inégales entre versions web et mobile.
        \item Interface parfois complexe ou non adaptée à l’usage mobile rapide.
        \item Problèmes de synchronisation ou de perte de données occasionnelle.
        \item Dépendance à la connexion internet pour certaines fonctions.
        \item Gestion des historiques ou des versions parfois limitée.
    \end{itemize}
\end{itemize}

\subsection{Organisation et archivage}
\begin{itemize}
    \item Utilisation de dossiers, tags, favoris, et moteurs de recherche pour classer et retrouver l’information.
    \item Archivage manuel ou automatisé, mais risque de fragmentation et de perte de contexte.
    \item Difficulté à maintenir une organisation cohérente sur le long terme.
\end{itemize}

\section{Défis et limitations de l’existant}

Malgré la richesse des outils disponibles, plusieurs défis majeurs persistent :
\begin{itemize}
    \item \textbf{Efficacité limitée :} La prise de notes rapide et structurée reste difficile, en particulier lors de situations de mobilité ou de multitâche.
    \item \textbf{Précision et fiabilité :} Les erreurs de transcription, la perte de données ou les problèmes de synchronisation peuvent nuire à la confiance dans l’outil.
    \item \textbf{Fragmentation de l’information :} La multiplication des supports et applications conduit à une dispersion des notes, rendant leur gestion et leur recherche complexes.
    \item \textbf{Accessibilité et ergonomie :} Les interfaces ne sont pas toujours adaptées à tous les usages (mobile, desktop, voix), ce qui limite l’adoption ou l’efficacité.
    \item \textbf{Confidentialité et dépendance :} La dépendance à des services cloud externes soulève des questions de confidentialité et de pérennité des données.
\end{itemize}

\section{Problématique}

Face à ces limites, de nombreux utilisateurs expriment le besoin d’une solution qui soit à la fois rapide, fiable, accessible sur tous leurs appareils, et capable d’organiser l’information sans effort supplémentaire. La question centrale devient alors : comment concevoir un outil de prise de notes qui réponde réellement aux attentes de mobilité, de fiabilité, d’ergonomie et de sécurité des utilisateurs modernes ?


\section*{Conclusion}
\addcontentsline{toc}{section}{Conclusion}
En conclusion, cette étude de l’existant révèle des besoins réels et non totalement satisfaits en matière de prise de notes et de gestion d’informations. L'analyse des pratiques et outils actuels a mis en évidence leurs forces et faiblesses, soulignant l'importance d'une réflexion approfondie pour répondre aux attentes des utilisateurs. Ce panorama complet constitue le socle sur lequel s’appuiera la conception d’une solution innovante, présentée dans les chapitres suivants.

% --- Chapter 2: Description of the proposed solution ---
\chapter{Description de la solution proposée}
% ===============================
% Chapter: Description de la solution proposée
% ===============================

\section*{Introduction}
\addcontentsline{toc}{section}{Introduction}
Cette section présente la solution SayNote, conçue pour révolutionner la prise de notes grâce à la voix, l’intelligence artificielle et une expérience utilisateur moderne.

\section{Présentation générale de SayNote}
SayNote est une application multiplateforme (web et mobile) dédiée à la prise de notes structurées, à l’édition intuitive et à la productivité augmentée. Elle s’adresse aux étudiants, professionnels et créateurs de contenu souhaitant capturer, organiser et partager leurs idées de manière fluide et efficace.

Après cette présentation générale, le diagramme de cas d’utilisation principal ci-dessous illustre les interactions majeures entre l’utilisateur et l’application :

\section{Fonctionnalités principales}
\begin{itemize}
    \item \textbf{Prise de notes et édition par la voix :} Créez, structurez et modifiez vos notes simplement en parlant.
    \item \textbf{Assistant IA intégré :} Posez des questions, obtenez des résumés, suggestions ou reformulations instantanées.
    \item \textbf{Éditeur par blocs flexible :} Organisez vos contenus en blocs modulables, avec une hiérarchie infinie et des options de formatage avancées.
    \item \textbf{Synchronisation temps réel et hors-ligne :} Accédez à vos notes sur tous vos appareils, même sans connexion internet.
    \item \textbf{Recherche et partage instantanés :} Trouvez n’importe quelle information en quelques secondes et partagez vos notes en un clic.
    \item \textbf{Personnalisation avancée :} Thèmes clair/sombre, icônes, mise en page adaptable à vos préférences.
\end{itemize}

Pour illustrer l’expérience utilisateur, la maquette suivante donne un aperçu de l’interface principale de SayNote :

\section{Architecture technique (aperçu)}
SayNote repose sur une architecture cloud moderne, combinant une interface web (Next.js/React), une application mobile (React Native/Expo), et un backend sécurisé (Supabase, PostgreSQL). L’intégration de l’API Gemini permet une transcription vocale précise et contextuelle.

Pour conclure cette section, le schéma ci-dessous présente une vue d’ensemble de l’architecture technique de la solution :

\begin{figure}[H]
    \centering
    \includegraphics[width=0.8\textwidth]{assets/docs/global_architecture.png}
    \caption{Schéma d’architecture globale de SayNote}
    \label{fig:global-architecture}
\end{figure}

\section{Cas d’utilisation principaux}
SayNote répond à une variété de scénarios d’utilisation courants :
\begin{itemize}
    \item \textbf{Prise de notes en réunion :} Capturer rapidement les points clés, décisions et actions lors de réunions professionnelles ou associatives.
    \item \textbf{Notes d’étude :} Structurer des résumés de cours, fiches de révision et annotations pour les étudiants.
    \item \textbf{Saisie d’idées spontanées :} Enregistrer une idée, une inspiration ou une tâche à tout moment grâce à la saisie vocale instantanée.
    \item \textbf{Édition collaborative :} Partager et modifier des notes en temps réel avec d’autres utilisateurs pour faciliter le travail d’équipe.
    \item \textbf{Gestion de listes et d’organisateurs personnels :} Créer des listes de tâches, agendas ou carnets de bord personnalisés.
\end{itemize}

\section{Sécurité et confidentialité}
La sécurité et la confidentialité des données sont au cœur de la conception de SayNote :
\begin{itemize}
    \item \textbf{Authentification sécurisée :} Utilisation de Supabase pour une gestion robuste des comptes et des accès.
    \item \textbf{Chiffrement des données :} Les notes et informations sensibles sont stockées de manière chiffrée côté serveur.
    \item \textbf{Respect de la vie privée :} Aucune donnée personnelle n’est partagée avec des tiers sans consentement explicite.
    \item \textbf{Conformité RGPD :} Les utilisateurs peuvent exporter ou supprimer leurs données à tout moment, conformément aux exigences européennes.
\end{itemize}

\section{Accessibilité et inclusivité}
SayNote vise à offrir une expérience accessible à tous les utilisateurs :
\begin{itemize}
    \item \textbf{Commandes vocales avancées :} Navigation et édition complètes via la voix, réduisant la dépendance au clavier ou à l’écran tactile.
    \item \textbf{Mode hors-ligne :} Accès et modification des notes sans connexion internet, avec synchronisation automatique lors du retour en ligne.
    \item \textbf{Interface adaptable :} Thèmes clair/sombre, tailles de police ajustables et contrastes optimisés pour les personnes malvoyantes.
    \item \textbf{Compatibilité multi-appareils :} Utilisation fluide sur ordinateurs, tablettes et smartphones.
\end{itemize}

\section{Comparaison avec les solutions existantes}
Bien que des applications comme Notion ou Evernote proposent des fonctionnalités avancées, SayNote se distingue par :
\begin{itemize}
    \item \textbf{Intégration native de la saisie et édition vocale :} La voix est au centre de l’expérience, permettant une prise de notes plus rapide et naturelle.
    \item \textbf{Assistant IA embarqué :} Génération de résumés, réponses contextuelles et suggestions automatiques directement dans l’application.
    \item \textbf{Simplicité d’usage :} Interface épurée, prise en main rapide et fonctionnalités essentielles accessibles en un clic.
    \item \textbf{Synchronisation temps réel et mode hors-ligne complet :} Continuité de l’expérience utilisateur, même sans réseau.
\end{itemize}

\section{Limites actuelles et évolutions prévues}
Malgré ses atouts, SayNote présente certaines limites actuelles :
\begin{itemize}
    \item \textbf{Reconnaissance vocale perfectible :} La précision peut varier selon l’accent ou l’environnement sonore ; des améliorations sont prévues via l’intégration de nouveaux modèles.
    \item \textbf{Fonctionnalités collaboratives en développement :} Les options de coédition et de gestion avancée des partages seront enrichies dans les prochaines versions.
    \item \textbf{Personnalisation limitée :} De nouveaux thèmes, widgets et options de configuration sont à l’étude pour mieux répondre aux besoins variés des utilisateurs.
    \item \textbf{Intégrations tierces :} L’ajout de connecteurs avec d’autres outils (calendriers, gestion de tâches, etc.) est planifié.
\end{itemize}

\section*{Conclusion}
\addcontentsline{toc}{section}{Conclusion}
SayNote propose une solution innovante et complète pour la prise de notes moderne, alliant puissance technologique, simplicité d’utilisation et adaptabilité aux besoins des utilisateurs.


% ===============================
% PART II (unnamed): Contribution
% ===============================

% --- Chapter 1: Conception (includes Visual Identity) ---
% --- PART II: Contribution ---
\cleardoublepage
\part*{Part II : Contribution}
\addcontentsline{toc}{part}{Part II : Contribution}
\thispagestyle{empty}

\chapter{Conception}
% ===============================
% Conception (inclut Identité Visuelle)
% ===============================

% --- Introduction non numérotée ---
\begin{center}
\textbf{\large Introduction du Chapitre}
\end{center}

\noindent
Ce chapitre présente la démarche de conception de VoiceNotion, en détaillant la méthodologie de planification, la conception UX, la modélisation, ainsi que l'identité visuelle de la marque. L'objectif est d'expliquer comment les choix de conception soutiennent la vision, l'expérience utilisateur et la cohérence de la marque.

% ===============================
% Section 1: Planification et UX
% ===============================
\section{Planification et conception UX}
% VoiceNotion - Chapitre II: Planification et Conception UX
% Ce chapitre présente la planification du projet et la conception de l'expérience utilisateur



\section{Introduction}

Après avoir identifié les problématiques et défini le concept de VoiceNotion dans le chapitre précédent, nous nous concentrons maintenant sur la planification du projet et la conception de l'expérience utilisateur. Cette phase est cruciale pour transformer notre vision en un plan d'action concret et en une interface utilisateur intuitive qui répond aux besoins réels des utilisateurs.

La planification et la conception UX sont particulièrement importantes pour VoiceNotion en raison de son approche novatrice combinant commandes vocales et édition par blocs. L'interaction vocale, bien que naturelle pour la communication humaine, présente des défis uniques lorsqu'elle est appliquée au contrôle d'une interface numérique. Notre objectif est de créer une expérience fluide et intuitive qui tire pleinement parti des capacités vocales tout en offrant une interface visuelle cohérente et familière.

Ce chapitre détaille notre méthodologie de planification, notre approche de l'expérience utilisateur basée sur la recherche et l'empathie, ainsi que la conception technique du système d'information qui sous-tend l'application.

\section{Planification du projet}

\subsection{Méthodologie de planification}

Pour le développement de VoiceNotion, nous avons adopté une approche agile, plus précisément la méthodologie Scrum, complétée par des éléments de Design Thinking pour la conception UX. Cette combinaison nous permet d'itérer rapidement tout en gardant l'utilisateur au centre de notre processus de conception.

\subsubsection{Approche Agile Scrum}

L'approche Scrum a été choisie pour sa flexibilité et sa capacité à s'adapter aux changements inhérents au développement d'un produit innovant comme VoiceNotion. Nous avons organisé notre travail en sprints de deux semaines, chacun comportant les événements standards de Scrum:

\begin{itemize}
    \item \textbf{Sprint Planning:} Au début de chaque sprint, l'équipe sélectionne les tâches du Product Backlog à réaliser, en se basant sur leur priorité et la capacité de l'équipe.
    
    \item \textbf{Daily Stand-up:} Réunions quotidiennes de 15 minutes pour synchroniser les activités et identifier les obstacles.
    
    \item \textbf{Sprint Review:} À la fin de chaque sprint, l'équipe présente les fonctionnalités développées aux parties prenantes pour obtenir des retours.
    
    \item \textbf{Sprint Retrospective:} L'équipe analyse ce qui a bien fonctionné et ce qui peut être amélioré pour le prochain sprint.
\end{itemize}

Cette approche nous permet de livrer régulièrement des incréments de produit fonctionnels et d'ajuster notre direction en fonction des retours utilisateurs et des défis techniques rencontrés.

\subsubsection{Design Thinking pour l'UX}

En parallèle de Scrum, nous avons intégré le processus de Design Thinking pour la conception de l'expérience utilisateur, qui comprend cinq phases:

\begin{enumerate}
    \item \textbf{Empathie:} Comprendre profondément les besoins, les frustrations et les aspirations des utilisateurs potentiels à travers des entretiens et des observations.
    
    \item \textbf{Définition:} Synthétiser les connaissances acquises pour définir clairement les problèmes à résoudre.
    
    \item \textbf{Idéation:} Générer un large éventail d'idées et de solutions potentielles sans contraintes initiales.
    
    \item \textbf{Prototypage:} Créer des représentations tangibles des solutions les plus prometteuses.
    
    \item \textbf{Test:} Recueillir les retours des utilisateurs sur les prototypes pour affiner et améliorer les solutions.
\end{enumerate}

Cette approche centrée sur l'utilisateur est particulièrement pertinente pour VoiceNotion, où l'interaction vocale nécessite une compréhension fine des attentes et des comportements des utilisateurs.

\begin{figure}[H]
    \centering
    %\includegraphics[width=0.8\textwidth]{assets/docs/agile_design_thinking.png}
    \caption{Intégration de Scrum et Design Thinking dans notre méthodologie}
    \label{fig:agile_design_thinking}
\end{figure}

\subsection{Planification}

\subsubsection{Roadmap du projet}

La roadmap de VoiceNotion a été structurée en quatre phases principales, chacune comportant plusieurs sprints:

\begin{enumerate}
    \item \textbf{Phase de recherche et de conception (2 mois):}
    \begin{itemize}
        \item Recherche utilisateur et analyse concurrentielle
        \item Définition des personas et des scénarios d'utilisation
        \item Conception de l'architecture du système
        \item Wireframing et prototypage initial
    \end{itemize}
    
    \item \textbf{Phase de développement MVP (4 mois):}
    \begin{itemize}
        \item Mise en place de l'infrastructure technique
        \item Développement du backend et de l'API
        \item Implémentation de l'éditeur de notes par blocs (avec BlockNote.js)
        \item Intégration de la reconnaissance vocale de base
        \item Développement de l'interface utilisateur mobile (Expo/React Native)
    \end{itemize}
    
    \item \textbf{Phase d'amélioration et d'expansion (3 mois):}
    \begin{itemize}
        \item Optimisation des algorithmes de traitement du langage naturel
        \item Expansion des commandes vocales supportées
        \item Développement de l'interface web
        \item Implémentation des fonctionnalités de sous-pages et de hiérarchie de notes
    \end{itemize}
    
    \item \textbf{Phase de finalisation et de lancement (1 mois):}
    \begin{itemize}
        \item Tests utilisateurs approfondis
        \item Correction des bugs et optimisations finales
        \item Préparation du matériel marketing
        \item Lancement officiel de l'application
    \end{itemize}
\end{enumerate}

\subsubsection{Gestion des risques}

Nous avons identifié plusieurs risques potentiels pour le projet et élaboré des stratégies d'atténuation:

\begin{table}[H]
\centering
\begin{tabular}{|p{3cm}|p{5cm}|p{5cm}|}
\hline
\textbf{Risque} & \textbf{Impact potentiel} & \textbf{Stratégie d'atténuation} \\
\hline
Précision limitée de la reconnaissance vocale & Frustration utilisateur, abandon de l'application & Implémentation d'un mécanisme de correction, modes d'entrée alternatifs, tests extensifs avec différents accents \\
\hline
Complexité technique de l'intégration BlockNote & Retards de développement, problèmes de performance & Spike techniques précoces, exploration des alternatives, recrutement d'expertise spécifique \\
\hline
Expérience utilisateur non intuitive & Courbe d'apprentissage abrupte, faible adoption & Tests utilisateurs fréquents, approche itérative, tutoriels intégrés \\
\hline
Limites des API Gemini & Fonctionnalités restreintes, dépendance à un tiers & Plan de secours avec solutions alternatives, découplage de l'architecture \\
\hline
Problèmes de performance sur appareils mobiles & Lenteur, consommation excessive de batterie & Optimisation continue, tests sur différents appareils, métriques de performance \\
\hline
\end{tabular}
\caption{Tableau des risques et stratégies d'atténuation}
\label{tab:risk_management}
\end{table}

\section{Expérience d'utilisateur (UX)}

\subsection{Recherche}

La conception de VoiceNotion est fondée sur une recherche approfondie pour comprendre les besoins, les attentes et les points de friction des utilisateurs potentiels en matière de prise de notes.

\subsubsection{Méthodologie de recherche}

Notre recherche a combiné plusieurs approches:

\begin{itemize}
    \item \textbf{Entretiens qualitatifs:} Nous avons mené 15 entretiens approfondis avec des utilisateurs potentiels issus de nos groupes cibles (étudiants, professionnels, créateurs de contenu).
    
    \item \textbf{Analyse concurrentielle:} Étude détaillée des applications existantes (Notion, Evernote, OneNote, Google Keep) pour identifier les forces, les faiblesses et les opportunités d'innovation.
    
    \item \textbf{Sondage en ligne:} Un questionnaire distribué à 150 participants pour quantifier les préférences et les habitudes de prise de notes.
    
    \item \textbf{Sessions d'observation:} Observation de 8 utilisateurs dans leur environnement naturel pendant qu'ils prenaient des notes, révélant des comportements et des défis non exprimés lors des entretiens.
\end{itemize}

\subsubsection{Principaux insights}

Cette recherche a mis en lumière plusieurs insights clés qui ont guidé notre conception:

\begin{enumerate}
    \item 78\% des participants trouvent que la saisie manuelle limite leur capacité à capturer rapidement les informations lors de réunions ou de conférences.
    
    \item Les utilisateurs de Notion apprécient la flexibilité de la structure par blocs, mais 65\% trouvent la courbe d'apprentissage trop abrupte.
    
    \item 92\% des participants ont exprimé de l'intérêt pour les commandes vocales, mais 71\% craignent qu'elles ne soient pas assez précises ou intuitives.
    
    \item Les utilisateurs mobiles (81\% de notre échantillon) prennent des notes dans des contextes variés où le clavier n'est pas toujours optimal (transports, déplacements, exercice).
    
    \item La structuration post-capture est identifiée comme l'un des plus grands défis, avec 85\% des participants qui admettent ne jamais réorganiser leurs notes brutes par manque de temps.
\end{enumerate}

\begin{figure}[H]
    \centering
    \includegraphics[width=0.75\textwidth]{assets/docs/user_research_insights.png}
    \caption{Synthèse des principaux insights de recherche utilisateur}
    \label{fig:user_research_insights}
\end{figure}

\subsection{Empathie}

\subsubsection{Personas utilisateur}

Basés sur notre recherche, nous avons développé trois personas principaux qui représentent nos utilisateurs cibles:

\paragraph{Persona 1: Sophie l'Étudiante}

\begin{itemize}
    \item \textbf{Profil:} 22 ans, étudiante en master de droit, utilise principalement son smartphone et son ordinateur portable pour prendre des notes.
    \item \textbf{Objectifs:} Capturer efficacement les informations en cours, organiser ses révisions, créer des liens entre différents concepts juridiques.
    \item \textbf{Frustrations:} Difficulté à suivre le rythme des professeurs, perd du temps à reformater ses notes, trouve la navigation entre ses documents fastidieuse.
    \item \textbf{Comportements:} Prend des notes pendant les cours, les complète en bibliothèque, révise régulièrement avec des mind maps et des fiches synthétiques.
    \item \textbf{Citation:} "Je passe plus de temps à essayer de noter tout ce que dit le prof qu'à vraiment comprendre le cours."
\end{itemize}

\paragraph{Persona 2: Marc le Professionnel}

\begin{itemize}
    \item \textbf{Profil:} 38 ans, chef de projet dans une entreprise technologique, toujours en déplacement entre réunions et sites clients.
    \item \textbf{Objectifs:} Capturer rapidement les décisions et actions des réunions, organiser ses projets, partager les informations avec son équipe.
    \item \textbf{Frustrations:} Manque de temps pour prendre des notes détaillées, difficultés à capturer des idées en déplacement, oublie des détails importants.
    \item \textbf{Comportements:} Utilise son téléphone pour des notes rapides, son ordinateur portable pour des documents plus structurés, souvent en multitâche.
    \item \textbf{Citation:} "Entre deux réunions, je n'ai parfois que 5 minutes pour noter les points clés avant d'enchaîner."
\end{itemize}

\paragraph{Persona 3: Leila la Créatrice de Contenu}

\begin{itemize}
    \item \textbf{Profil:} 29 ans, auteure et créatrice de contenu indépendante, travaille depuis chez elle ou dans des cafés.
    \item \textbf{Objectifs:} Capturer l'inspiration quand elle survient, structurer ses idées en articles ou en scripts, organiser ses recherches.
    \item \textbf{Frustrations:} Perd ses meilleures idées quand elle ne peut pas les noter immédiatement, passe trop de temps à organiser son contenu, lutte avec différents formats de notes.
    \item \textbf{Comportements:} Alterne entre notes vocales, notes manuscrites et documents numériques, travaille de manière non linéaire avec beaucoup d'itérations.
    \item \textbf{Citation:} "Mes meilleures idées me viennent souvent quand je suis loin de mon ordinateur, pendant une promenade ou sous la douche."
\end{itemize}

\begin{figure}[H]
    \centering
    %\includegraphics[width=0.9\textwidth]{assets/docs/voicenotion_personas.png}
    \caption{Les trois personas principaux de VoiceNotion}
    \label{fig:voicenotion_personas}
\end{figure}

\subsubsection{Scénarios d'utilisations}

Pour chaque persona, nous avons développé des scénarios d'utilisation qui illustrent comment VoiceNotion répondrait à leurs besoins spécifiques:

\paragraph{Scénario 1: Sophie en cours de droit constitutionnel}

Sophie assiste à un cours de droit constitutionnel où le professeur parle rapidement et fait référence à de nombreux articles et précédents. Avec VoiceNotion, elle:
\begin{enumerate}
    \item Active l'application et commence à prendre des notes textuelles de base.
    \item Utilise la commande vocale "Nouveau titre: Articles constitutionnels importants" pour créer une section structurée sans interrompre sa prise de notes.
    \item Dit "Ajouter liste à puces" pour commencer une liste des articles mentionnés.
    \item Alterne facilement entre la saisie vocale pour les concepts généraux et la saisie manuelle pour les termes techniques précis ou les références.
    \item À la fin du cours, utilise la commande "Créer une sous-page pour les cas jurisprudentiels" pour organiser les exemples mentionnés dans une structure hiérarchique.
\end{enumerate}

\paragraph{Scénario 2: Marc en réunion client puis en déplacement}

Marc participe à une réunion importante avec un client pour discuter des spécifications d'un nouveau projet:
\begin{enumerate}
    \item Au début de la réunion, il ouvre VoiceNotion et crée une nouvelle note avec la structure de base (objectifs, spécifications, actions).
    \item Pendant la discussion, il ajoute rapidement des points à chaque section avec des commandes vocales discrètes.
    \item Il utilise la commande "Ajouter liste à puces: Points à suivre" pour créer une liste d'actions à réaliser.
    \item Après la réunion, dans le taxi, il révise ses notes et utilise la commande vocale "Réorganiser: déplacer la section Budget après Échéancier" pour restructurer son document.
    \item Il crée une sous-page pour les détails techniques qui nécessitent une exploration plus approfondie.
\end{enumerate}

\paragraph{Scénario 3: Leila trouve l'inspiration pendant une promenade}

Leila fait une promenade quotidienne quand elle a une idée pour un nouvel article:
\begin{enumerate}
    \item Elle sort son téléphone et ouvre VoiceNotion, puis dit "Nouvelle note: Idée d'article sur la créativité et la routine".
    \item En marchant, elle dicte ses idées principales, utilisant des commandes comme "Nouveau paragraphe" ou "Point important" pour structurer sa pensée.
    \item Elle dit "Ajouter référence: livre Flow de Mihaly Csikszentmihalyi" pour ne pas oublier cette source.
    \item De retour chez elle, elle reprend la note sur son ordinateur, où elle peut voir la structure déjà organisée et commencer à développer chaque section.
    \item Elle utilise la fonctionnalité de bloc toggle pour cacher certaines sections et se concentrer sur l'introduction qu'elle rédige maintenant manuellement.
\end{enumerate}

\section{Conception du système d'information}

\subsection{Identification des acteurs}

Le système VoiceNotion interagit avec plusieurs types d'acteurs, chacun ayant des objectifs et des interactions spécifiques:

\begin{itemize}
    \item \textbf{Utilisateur non authentifié:} Peut explorer la landing page, créer un compte ou se connecter.
    
    \item \textbf{Utilisateur authentifié:} Le principal acteur du système, qui peut créer, modifier, organiser et exporter des notes.
    
    \item \textbf{API Gemini:} Acteur système externe qui traite les commandes vocales et retourne des instructions structurées.
    
    \item \textbf{Service de stockage:} Acteur système responsable de la persistance et de la synchronisation des données.
\end{itemize}

\subsection{Diagramme de cas d'utilisation}

Le diagramme de cas d'utilisation ci-dessous illustre les principales interactions entre les acteurs et le système VoiceNotion:

\begin{figure}[H]
    \centering
    \includegraphics[width=0.9\textwidth]{assets/docs/voicenotion_use_case.png}
    \caption{Diagramme de cas d'utilisation pour VoiceNotion}
    \label{fig:use_case_diagram}
\end{figure}

Les principaux cas d'utilisation incluent:

\begin{itemize}
    \item \textbf{Gestion du compte:} Inscription, connexion, modification du profil.
    
    \item \textbf{Gestion des notes:} Création, édition, suppression, création de sous-pages.
    
    \item \textbf{Interaction vocale:} Dictée de contenu, commandes de formatage, navigation par la voix.
    
    \item \textbf{Édition structurée:} Manipulation des blocs, formatage du texte, insertion d'éléments riches.
    
    \item \textbf{Recherche et filtrage:} Recherche textuelle, filtrage par date.
    
    \item \textbf{Exportation:} Export des notes vers différents formats.
\end{itemize}

\subsection{Diagrammes de séquence}

Pour illustrer les interactions dynamiques entre l'utilisateur, l'application et les services externes, nous avons créé des diagrammes de séquence pour les processus clés.

\subsubsection{Séquence de commande vocale}

Le diagramme suivant montre la séquence d'interactions lors de l'utilisation d'une commande vocale pour manipuler le contenu:

\begin{figure}[H]
    \centering
    \includegraphics[width=0.85\textwidth]{assets/docs/voicenotion_sequence_voice.png}
    \caption{Diagramme de séquence pour le traitement d'une commande vocale}
    \label{fig:sequence_voice_command}
\end{figure}

\subsubsection{Séquence de création et sauvegarde de note}

Ce diagramme illustre le processus de création, d'édition et de sauvegarde d'une note:

\begin{figure}[H]
    \centering
    \includegraphics[width=0.85\textwidth]{assets/docs/voicenotion_sequence_save.png}
    \caption{Diagramme de séquence pour la création et sauvegarde d'une note}
    \label{fig:sequence_save_note}
\end{figure}

\subsubsection{Séquence d'authentification}

Ce diagramme illustre le processus d'authentification des utilisateurs dans l'application:

\begin{figure}[H]
    \centering
    \includegraphics[width=0.85\textwidth]{assets/docs/voicenotion_auth_sequence.png}
    \caption{Diagramme de séquence pour l'authentification}
    \label{fig:sequence_auth}
\end{figure}

\subsection{Diagrammes de base de données}

\subsubsection{Diagramme entité-relation}

Le modèle entité-relation ci-dessous représente la structure conceptuelle des données pour VoiceNotion:

\begin{figure}[H]
    \centering
    \includegraphics[width=0.9\textwidth]{assets/docs/voicenotion_er_diagram.png}
    \caption{Diagramme entité-relation pour VoiceNotion}
    \label{fig:er_diagram}
\end{figure}

Les principales entités et leurs relations sont:

\begin{itemize}
    \item \textbf{User:} Stocke les informations utilisateur (email, nom d'affichage, avatar).
    
    \item \textbf{Note:} L'entité centrale qui contient les métadonnées d'une note (titre, date de création/modification, propriétaire).
    
    \item \textbf{Block:} Représente un bloc individuel dans une note, avec son type, contenu et position.
    
    \item \textbf{Subpage:} Gère la relation hiérarchique entre les notes, permettant la création de sous-pages.
    
    \item \textbf{VoiceCommand:} Stocke les commandes vocales et leurs paramètres.
\end{itemize}

\subsubsection{Diagramme de base de données}

Le schéma logique de la base de données traduit le modèle entité-relation en une structure implémentable:

\begin{figure}[H]
    \centering
    \includegraphics[width=0.9\textwidth]{assets/docs/voicenotion_db_schema.png}
    \caption{Schéma logique de la base de données pour VoiceNotion}
    \label{fig:db_schema}
\end{figure}

Notre implémentation utilise une base de données PostgreSQL hébergée sur Supabase, avec les tables principales suivantes:

\begin{itemize}
    \item \textbf{users:} id, email, display\_name, avatar\_url, created\_at, updated\_at
    
    \item \textbf{notes:} id, title, user\_id, created\_at, updated\_at, is\_archived
    
    \item \textbf{blocks:} id, note\_id, parent\_block\_id, type, content, position, created\_at, updated\_at
    
    \item \textbf{subpages:} id, parent\_note\_id, child\_note\_id, position, created\_at
    
    \item \textbf{voice\_commands:} id, user\_id, command\_text, intent\_type, parameters, created\_at
\end{itemize}

\section{Conclusion}

Ce chapitre a présenté notre approche méthodique de la planification du projet VoiceNotion et de la conception de son expérience utilisateur. En combinant une méthodologie agile avec une approche de Design Thinking centrée sur l'utilisateur, nous avons établi un cadre solide pour le développement d'une application qui répond véritablement aux besoins des utilisateurs.

La recherche utilisateur approfondie et l'élaboration de personas détaillés nous ont permis de comprendre intimement les défis et les aspirations de nos utilisateurs cibles. Les scénarios d'utilisation ont illustré comment VoiceNotion s'intégrerait naturellement dans leurs flux de travail quotidiens, offrant une valeur réelle et des améliorations tangibles à leur expérience de prise de notes.

La conception technique du système, illustrée par les diagrammes de cas d'utilisation, de séquence et de base de données, fournit une base solide pour l'implémentation qui sera détaillée dans le chapitre suivant. Cette architecture a été conçue pour être robuste, évolutive et flexible, permettant d'accommoder les futures améliorations et extensions de l'application.

Les prochaines étapes consistent à transformer cette conception en un produit fonctionnel à travers le développement technique, en restant fidèle à notre vision d'une application de prise de notes révolutionnaire qui libère la créativité et la productivité de ses utilisateurs grâce à la puissance de la voix. 


% ===============================
% Section 2: Identité Visuelle
% ===============================
\section{Identité visuelle}
\section{VoiceNotion – Brand Identity Snapshot}

\subsection{Brand Essence}
\begin{quote}
    “A voice-first productivity companion that brings calm, clarity, and a spark of joy to capturing and organizing your thoughts.”
\end{quote}

\subsection{Core Brand Personality (4 Traits)}
\begin{tabular}{|l|l|}
\hline
\textbf{Trait} & \textbf{Description} \\
\hline
Calm & Minimal, distraction-free, focused like Notion. \\
\hline
Clear & Straightforward and precise—no clutter, no fluff. \\
\hline
Supportive & Friendly, helpful, and empowering—always welcoming. \\
\hline
Lightly Joyful & Adds subtle positivity and delight—never too serious. \\
\hline
\end{tabular}

\subsection{Positioning Pillars}
\begin{tabular}{|l|l|}
\hline
\textbf{Pillar} & \textbf{VoiceNotion Delivers} \\
\hline
100\% Voice-First & Speak to write, format, and organize—hands-free productivity \\
\hline
AI-Powered Clarity & Smart transcription, summaries, and auto-tagging \\
\hline
Calm \& Focused UX & Thoughtful, minimal interface to reduce friction \\
\hline
Inclusive Design & Accessibility-first, multilingual, voice-only navigation \\
\hline
\end{tabular}

\subsection{Key Differentiators}
\begin{tabular}{|l|l|l|}
\hline
\textbf{Feature} & \textbf{Others} & \textbf{VoiceNotion} \\
\hline
Fully voice-operated & \ding{55} & \ding{51} \\
\hline
No typing required & \ding{55} & \ding{51} \\
\hline
AI summaries + tagging & \ding{51} & \ding{51} \\
\hline
Voice-controlled workspace & \ding{55} & \ding{51} \\
\hline
Accessibility-first UX & \ding{55} & \ding{51} \\
\hline
\end{tabular}

\subsection{Brand Voice Principles}
\begin{itemize}
    \item \textbf{Calm \& Clear}: “Start speaking and I’ll take it from here.”
    \item \textbf{Supportive \& Positive}: “Got it! Note saved.”
    \item \textbf{Simple, No Jargon}: “Your voice. Your notes. No keyboard.”
    \item \textbf{No Talking Back}: Text-based feedback only. Quiet UX.
\end{itemize}

\section{Visual Identity System – VoiceNotion}

\subsection{Accent Color Suggestion}
\subsubsection{Soft Indigo (\#5C6AC4)}
\begin{itemize}
    \item \textbf{Why?}: Indigo has a calm, thoughtful energy, but this soft variant adds a \textbf{modern, focused, yet gently vibrant} feel.
    \item It’s professional \textbf{without being too corporate}, and joyful \textbf{without being childish}.
    \item Also works beautifully with dark mode AND light mode.
\end{itemize}

\subsubsection{Alternatives (in same vibe):}
\begin{itemize}
    \item \textbf{Slate Blue} \#6A7FDB – more techy \& cool
    \item \textbf{Muted Orchid} \#A88FBD – adds subtle joy + uniqueness
    \item \textbf{Cool Mint} \#9EEBCF – fresh, accessible, voice-inspired
\end{itemize}

% Placeholder for Color Palette Image
% \begin{figure}[h!]
%     \centering
%     \includegraphics[width=0.8\textwidth]{assets/docs/colors/color_palette.png}
%     \caption{VoiceNotion Color Palette}
%     \label{fig:color_palette}
% \end{figure}

\subsection{Font Suggestions}
You want something clean, modern, and human — just like your brand tone.
\begin{tabular}{|l|l|l|}
\hline
\textbf{Font} & \textbf{Style} & \textbf{Why It Fits} \\
\hline
\textbf{Inter} & Sans-serif, modern & Designed for interfaces. Clean, readable, calm vibe. \\
\hline
\textbf{General Sans} & Friendly sans-serif & Slightly rounded for a friendly, clear feel. \\
\hline
\textbf{Outfit} & Geometric, simple & Soft and techy. Great for voice interfaces. \\
\hline
\textbf{Satoshi} & Neutral + modern & Balanced between formal and casual. Great for clarity. \\
\hline
\end{tabular}

\textit{Avoid overly playful fonts like Comic Neue or too serious fonts like Times New Roman.}

\subsection{Visual Elements and Style}

\subsubsection{Iconography}
\begin{itemize}
    \item Use \textbf{thin-line, rounded icons} (e.g., Lucide, Feather Icons)
    \item Avoid overly detailed or harsh outlines
\end{itemize}

\subsubsection{Shapes}
\begin{itemize}
    \item Rounded corners (8px–16px) for buttons, cards, inputs
    \item Fluid shapes to suggest voice flow (like audio waves, soft curves)
\end{itemize}

\subsubsection{Imagery \& Illustrations}
\begin{itemize}
    \item Use \textbf{abstract voice waveforms}, \textbf{flow lines}, and \textbf{floating notes}
    \item Keep imagery light and clean (no overly literal or corporate stock)
\end{itemize}

\subsubsection{Themes}
\begin{itemize}
    \item \textbf{Light Mode}: Soft off-white background (\#FAFAFC)
    \item \textbf{Dark Mode}: Deep indigo/blue-gray (\#1E1E2F) with your accent color glowing subtly
\end{itemize}

\subsection{Summary Visual System}
\begin{tabular}{|l|l|}
\hline
\textbf{Element} & \textbf{Design Direction} \\
\hline
\textbf{Accent Color} & Soft Indigo \#5C6AC4 \\
\hline
\textbf{Typography} & Inter / General Sans / Outfit \\
\hline
\textbf{UI Style} & Minimal, rounded, lightly playful \\
\hline
\textbf{Icon Style} & Line-based, round corners, consistent weight \\
\hline
\textbf{Backgrounds} & Light: \#FAFAFC / Dark: \#1E1E2F \\
\hline
\textbf{Mood} & Calm, voice-centric, modern, slightly joyful \\
\hline
\end{tabular}


% --- Conclusion non numérotée ---
\vspace{1cm}
\begin{center}
\textbf{\large Conclusion du Chapitre}
\end{center}

\noindent
Ce chapitre a mis en lumière l'importance d'une conception réfléchie, de la planification à l'identité visuelle. L'intégration de l'UX et du branding garantit une expérience utilisateur optimale et une marque forte, éléments essentiels pour le succès et la différenciation du projet VoiceNotion.



% --- Chapter 2: Implementation ---
\chapter{Implémentation}
\input{saynote_implementation}

% --- General Conclusion ---
\section*{Conclusion générale}
\addcontentsline{toc}{section}{Conclusion générale}
% ===============================
% General Conclusion
% ===============================
\chapter*{Conclusion générale}
\addcontentsline{toc}{chapter}{Conclusion générale}
Ce mémoire a présenté le processus de conception, de développement et d’implémentation de l’application VoiceNotion. Nous avons analysé les limites des méthodes traditionnelles de prise de notes, défini une solution innovante fondée sur la reconnaissance vocale et l’édition par blocs, puis détaillé la planification, la conception UX, l’identité visuelle et l’implémentation technique.

VoiceNotion se distingue par sa capacité à rendre la prise de notes plus rapide, accessible et flexible pour un large éventail d’utilisateurs. L’architecture technique moderne, l’accent mis sur l’expérience utilisateur et la sécurité des données, ainsi que l’intégration d’une identité visuelle cohérente, constituent les points forts du projet.

Les perspectives d’évolution incluent l’amélioration continue de la reconnaissance vocale, l’ajout de fonctionnalités collaboratives et l’optimisation des performances. Ce travail pose ainsi les bases d’une solution évolutive, adaptée aux besoins futurs en matière de productivité et de gestion de l’information.


% --- Online Resources ---
\chapter*{Ressources en ligne}
\addcontentsline{toc}{chapter}{Ressources en ligne}
Voici une liste non exhaustive de documents et articles de référence consultés durant ce travail :

\begin{enumerate}[label=\arabic*., leftmargin=*]
  \item \href{https://supabase.com/docs}{Supabase Documentation}\label{res:supabase}
  \item \href{https://nextjs.org/docs}{Next.js Documentation}\label{res:nextjs}
  \item \href{https://reactnative.dev/docs/getting-started}{React Native Docs}\label{res:reactnative}
  \item \href{https://docs.expo.dev}{Expo Documentation}\label{res:expo}
  \item \href{https://docs.netlify.com}{Netlify Docs}\label{res:netlify}
  \item \href{https://www.postgresql.org/docs/}{PostgreSQL Official Docs}\label{res:postgresql}
  \item \href{https://figma.com/resources/learn-design}{Figma Design Guides}\label{res:figma}
  \item \href{https://jestjs.io/docs}{Jest Testing Framework Docs}\label{res:jest}
  \item \href{https://docs.cypress.io}{Cypress End-to-End Testing}\label{res:cypress}
  \item \href{https://www.typescriptlang.org/docs}{TypeScript Handbook}\label{res:typescript}
  \item \href{https://nodejs.org/en/docs}{Node.js Documentation}\label{res:node}
  \item \href{https://docs.github.com/en/actions}{GitHub Actions Docs}\label{res:githubactions}
  \item \href{https://jwt.io/introduction}{JWT.io – JSON Web Tokens}\label{res:jwt}
  \item \href{https://owasp.org/www-project-top-ten/}{OWASP Top Ten Security Risks}\label{res:owasp}
  \item \href{https://developer.mozilla.org/en-US/docs/Web/JavaScript/Guide}{MDN JavaScript Guide}\label{res:mdn}
  \item \href{https://cloud.google.com/architecture}{Google Cloud Architecture Center}\label{res:gcp}
  \item \href{https://eslint.org/docs/latest}{ESLint Documentation}\label{res:eslint}
  \item \href{https://prettier.io/docs/en/index.html}{Prettier Formatter Docs}\label{res:prettier}
  \item \href{https://tailwindcss.com/docs}{Tailwind CSS Docs}\label{res:tailwind}
  \item \href{https://testing-library.com/docs/react-testing-library/intro}{React Testing Library}\label{res:reacttestinglibrary}
  \item \href{https://developer.chrome.com/docs/lighthouse/}{Google Lighthouse Docs}\label{res:lighthouse}
  \item \href{https://wix.github.io/Detox/docs/api}{Detox Testing Docs}\label{res:detox}
  \item \href{https://yjs.dev/}{Yjs CRDT Framework}\label{res:crdt}
\end{enumerate}

\end{document}
