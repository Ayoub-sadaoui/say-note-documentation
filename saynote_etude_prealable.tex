% SayNote - Chapitre 1 : Étude de l’existant
% Ce chapitre analyse l’état actuel de la prise de notes et de la gestion d’informations, en s’appuyant sur les pratiques, outils et défis rencontrés par les utilisateurs.

\section*{Introduction}
\addcontentsline{toc}{section}{Introduction}
Ce chapitre dresse un état des lieux des pratiques actuelles de prise de notes et de gestion d’informations. Il identifie les méthodes utilisées, les difficultés rencontrées par les utilisateurs, et les limites des solutions existantes. L’objectif est de comprendre le contexte et les besoins réels des utilisateurs afin de mettre en lumière les défis à relever, constituant ainsi le socle de notre réflexion pour une solution innovante.

\section{Contexte de la prise de notes}

La prise de notes intervient dans des contextes variés : réunions, cours, brainstorming, déplacements, etc. Les utilisateurs cherchent à capturer rapidement des idées, à structurer des informations, et à pouvoir y accéder facilement sur différents supports (papier, ordinateur, mobile). L’évolution des usages montre une transition progressive du papier vers les outils numériques, sans pour autant résoudre tous les problèmes liés à la gestion efficace des notes.

\section{Méthodes actuelles de gestion des notes et informations}

La gestion des notes repose aujourd’hui sur une grande diversité de méthodes et d’outils, qu’on peut regrouper en plusieurs catégories :

\subsection{Prise de notes manuelle}
\begin{itemize}
    \item Utilisation de carnets, feuilles volantes, post-its.
    \item Facilité d’accès mais risque élevé de perte, de désorganisation ou d’oubli.
    \item Difficulté à rechercher ou à partager l’information.
\end{itemize}

\subsection{Outils numériques classiques}
\begin{itemize}
    \item Logiciels de traitement de texte (Word, Google Docs), tableurs, emails.
    \item Avantages : sauvegarde, partage, édition facile.
    \item Limites : structure peu adaptée à la prise de notes rapide, fragmentation des fichiers, manque de synchronisation multi-appareils.
\end{itemize}

\subsection{Applications spécialisées de prise de notes}
\begin{itemize}
    \item Applications comme Notion, Evernote, OneNote, Google Keep, Apple Notes.
    \item Fonctions avancées : organisation hiérarchique, recherche, étiquettes, synchronisation cloud, prise de notes vocale.
    \item Limitations relevées par les utilisateurs :
    \begin{itemize}
        \item Transcription vocale parfois imprécise, surtout en environnement bruyant ou multilingue.
        \item Fonctionnalités inégales entre versions web et mobile.
        \item Interface parfois complexe ou non adaptée à l’usage mobile rapide.
        \item Problèmes de synchronisation ou de perte de données occasionnelle.
        \item Dépendance à la connexion internet pour certaines fonctions.
        \item Gestion des historiques ou des versions parfois limitée.
    \end{itemize}
\end{itemize}

\subsection{Organisation et archivage}
\begin{itemize}
    \item Utilisation de dossiers, tags, favoris, et moteurs de recherche pour classer et retrouver l’information.
    \item Archivage manuel ou automatisé, mais risque de fragmentation et de perte de contexte.
    \item Difficulté à maintenir une organisation cohérente sur le long terme.
\end{itemize}

\section{Défis et limitations de l’existant}

Malgré la richesse des outils disponibles, plusieurs défis majeurs persistent :
\begin{itemize}
    \item \textbf{Efficacité limitée :} La prise de notes rapide et structurée reste difficile, en particulier lors de situations de mobilité ou de multitâche.
    \item \textbf{Précision et fiabilité :} Les erreurs de transcription, la perte de données ou les problèmes de synchronisation peuvent nuire à la confiance dans l’outil.
    \item \textbf{Fragmentation de l’information :} La multiplication des supports et applications conduit à une dispersion des notes, rendant leur gestion et leur recherche complexes.
    \item \textbf{Accessibilité et ergonomie :} Les interfaces ne sont pas toujours adaptées à tous les usages (mobile, desktop, voix), ce qui limite l’adoption ou l’efficacité.
    \item \textbf{Confidentialité et dépendance :} La dépendance à des services cloud externes soulève des questions de confidentialité et de pérennité des données.
\end{itemize}

\section{Problématique}

Face à ces limites, de nombreux utilisateurs expriment le besoin d’une solution qui soit à la fois rapide, fiable, accessible sur tous leurs appareils, et capable d’organiser l’information sans effort supplémentaire. La question centrale devient alors : comment concevoir un outil de prise de notes qui réponde réellement aux attentes de mobilité, de fiabilité, d’ergonomie et de sécurité des utilisateurs modernes ?


\section*{Conclusion}
\addcontentsline{toc}{section}{Conclusion}
En conclusion, cette étude de l’existant révèle des besoins réels et non totalement satisfaits en matière de prise de notes et de gestion d’informations. L'analyse des pratiques et outils actuels a mis en évidence leurs forces et faiblesses, soulignant l'importance d'une réflexion approfondie pour répondre aux attentes des utilisateurs. Ce panorama complet constitue le socle sur lequel s’appuiera la conception d’une solution innovante, présentée dans les chapitres suivants.