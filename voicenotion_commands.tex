% VoiceNotion Voice Commands Reference
% This file can be included in the main LaTeX document

\chapter{Voice Command Reference}

\section{Basic Voice Commands}

\begin{longtable}{p{0.25\textwidth}p{0.5\textwidth}p{0.25\textwidth}}
\toprule
\textbf{Command} & \textbf{Description} & \textbf{Example} \\
\midrule
\endhead

% Text Input Commands
"New note" & Creates a new note & "New note" \\
"Title [text]" & Sets the title of the current note & "Title meeting minutes" \\
"Add text [content]" & Adds a paragraph block with the specified text & "Add text Remember to follow up with the team" \\
\midrule

% Block Creation Commands
"New paragraph" & Creates a new paragraph block & "New paragraph" \\
"Heading [level] [text]" & Creates a heading of specified level (1-6) & "Heading 2 Project Goals" \\
"Bullet list" & Starts a bullet list & "Bullet list" \\
"Add bullet [text]" & Adds a bullet point to an existing list & "Add bullet Schedule meeting" \\
"Numbered list" & Starts a numbered list & "Numbered list" \\
"Add item [text]" & Adds an item to an existing numbered list & "Add item Contact suppliers" \\
"Checklist" & Creates a to-do list with checkboxes & "Checklist" \\
"Add task [text]" & Adds a task to an existing checklist & "Add task Submit report" \\
"Quote [text]" & Creates a block quote & "Quote This is important" \\
"Code block [language]" & Creates a code block with optional language specification & "Code block javascript" \\
"Toggle [title]" & Creates a toggle block with the specified title & "Toggle Implementation details" \\
\midrule

% Text Formatting Commands
"Bold [text]" & Makes selected text or next dictated text bold & "Bold important point" \\
"Italic [text]" & Makes selected text or next dictated text italic & "Italic for emphasis" \\
"Underline [text]" & Underlines selected text or next dictated text & "Underline deadline" \\
"Strikethrough [text]" & Applies strikethrough to selected text or next dictated text & "Strikethrough old version" \\
"Link [text] to [URL]" & Creates a hyperlink & "Link documentation to https://docs.example.com" \\
"Code [text]" & Formats text as inline code & "Code console.log()" \\
\midrule

% Selection and Navigation
"Select last paragraph" & Selects the most recently created paragraph & "Select last paragraph" \\
"Select line" & Selects the current line & "Select line" \\
"Select all" & Selects all content in the current note & "Select all" \\
"Go to start" & Moves cursor to the beginning of the note & "Go to start" \\
"Go to end" & Moves cursor to the end of the note & "Go to end" \\
\midrule

% Editing Commands
"Delete [selection]" & Deletes the specified selection & "Delete last paragraph" \\
"Clear selection" & Clears the current selection & "Clear selection" \\
"Replace [selection] with [text]" & Replaces specified text with new content & "Replace last sentence with updated information" \\
"Cut selection" & Cuts the current selection & "Cut selection" \\
"Copy selection" & Copies the current selection & "Copy selection" \\
"Paste" & Pastes copied content & "Paste" \\
\midrule

% Block Manipulation
"Move block up" & Moves the current block up one position & "Move block up" \\
"Move block down" & Moves the current block down one position & "Move block down" \\
"Indent block" & Indents the current block to create nesting & "Indent block" \\
"Outdent block" & Reduces indentation of the current block & "Outdent block" \\
"Convert to [block type]" & Converts the current block to a different type & "Convert to heading 3" \\
\midrule

% Note Management
"Save note" & Saves the current note & "Save note" \\
"Rename note to [name]" & Renames the current note & "Rename note to Project Ideas" \\
"Delete note" & Deletes the current note after confirmation & "Delete note" \\
"Share note" & Opens sharing options for the current note & "Share note" \\
"Export as [format]" & Exports the note in the specified format (PDF, HTML, etc.) & "Export as PDF" \\
\midrule

% Undo/Redo
"Undo" & Undoes the last action & "Undo" \\
"Redo" & Redoes the previously undone action & "Redo" \\

\bottomrule
\caption{VoiceNotion Voice Commands}
\label{tab:voice_commands}
\end{longtable}

\section{Advanced Voice Commands}

\begin{longtable}{p{0.25\textwidth}p{0.5\textwidth}p{0.25\textwidth}}
\toprule
\textbf{Command} & \textbf{Description} & \textbf{Example} \\
\midrule
\endhead

% Advanced Formatting
"Format [selection] as [style]" & Applies specified format to selection & "Format selection as bold italic" \\
"Set text color to [color]" & Changes text color & "Set text color to blue" \\
"Set background to [color]" & Changes background color of selection & "Set background to yellow" \\
\midrule

% Multi-step Commands
"Create section with title [title]" & Creates a section with a heading and empty paragraph & "Create section with title Literature Review" \\
"Add bullet list with [item1], [item2]" & Creates a bullet list with multiple items & "Add bullet list with apples, oranges, bananas" \\
"Insert table with [rows] rows and [columns] columns" & Creates a table with specified dimensions & "Insert table with 3 rows and 4 columns" \\
\midrule

% Media and Attachments
"Add image [description]" & Opens image selector with optional description & "Add image project logo" \\
"Insert link to file" & Opens file selector to link to a file & "Insert link to file" \\
"Embed video" & Opens video embedding interface & "Embed video" \\
\midrule

% Template Commands
"Apply template [template name]" & Applies a saved template to the current note & "Apply template meeting notes" \\
"Save as template named [name]" & Saves current note structure as a template & "Save as template named weekly report" \\
\midrule

% Organization Commands
"Add tag [tag name]" & Adds a tag to the current note & "Add tag important" \\
"Move to folder [folder name]" & Moves the note to the specified folder & "Move to folder Projects" \\
"Create folder [folder name]" & Creates a new folder & "Create folder Research" \\

\bottomrule
\caption{Advanced VoiceNotion Voice Commands}
\label{tab:advanced_commands}
\end{longtable}

\section{Context-Specific Commands}

The following commands are context-sensitive and their behavior depends on the current state of the editor.

\begin{longtable}{p{0.25\textwidth}p{0.75\textwidth}}
\toprule
\textbf{Command} & \textbf{Behavior} \\
\midrule
\endhead

"Continue" & Continues dictation at the current position \\
"Stop" & Stops the current dictation session \\
"Cancel" & Cancels the current command or operation \\
"Help" & Displays context-sensitive help for the current view \\
"What can I say?" & Shows available commands for the current context \\
"Show commands for [feature]" & Displays commands related to a specific feature \\

\bottomrule
\caption{Context-Specific Commands}
\label{tab:context_commands}
\end{longtable} 